% CHAPTER 1
\chapter{POWER SYSTEM FREQUENCY STABILITY}
\label{chp:2}

\section{Frequency in a Power System}
\subsection{Synchronous Generator and Synchronous Speed}

Synchronous machines produce torque only in synchronous speed. This is why they are equipped with damper windings which are basically induction machine windings. If the frequency of  grid changes, damper windings create a torque which creates a force to synchronize the speed to the grid frequency. Two type of damper windings are given in Figure \ref{damperwindings}.

\begin{figure}[h!]
	\centering
	\begin{subfigure}{.5\textwidth}
		\centering
		\includegraphics[width=.95\linewidth]{continuousdamper.png}
		\caption{continuous damper}
		\label{continuousdamper}
	\end{subfigure}%
	\begin{subfigure}{.5\textwidth}
		\centering
		\includegraphics[width=.95\linewidth]{noncontinuousdamper.png}
		\caption{non-continuous damper}
		\label{noncontinuousdamper}
	\end{subfigure}
	\caption{Damper windings in a synchronous generator \cite{Kundur}}
	\label{damperwindings}
\end{figure}

Due to the damper windings in the rotor, the synchronous machines always operate in synchronous speed. Relation between grid frequency and the synchronous speed is given in \ref{syncspeed}

\begin{equation}
 n_{s}=\frac{120f}{p_{f}}\cite{Kundur}
\end{equation}
\begin{equation}
 n_{s}=\frac{60}{2\pi}\omega_{syn}
\end{equation}
\begin{equation}
\omega_{syn}=\frac{4\pi f}{p_{f}}
\label{syncspeed}
\end{equation}

where $n_{s}$ is the synchronous speed in rpm, $f$ is the grid frequency in Hz, $p_{f}$ is the number of poles of the corresponding generator and $\omega_{syn}$ is the synchronous angular speed in rad/s.

\subsection{Swing Equation}
Speed in synchronous machines changes according to the net torque acting on the rotor. Therefore, the speed is maintained constant unless there is no difference between mechanical and electromechanical torque. The equation of motion is given in Eq.\ref{eqmotion} where $J$ is aggravated moment of inertia of the generator and the turbine in $kgm^{2}$,$T_{m}$ and $T_{e}$ are mechanical and electromechanical torques in $Nm$.

\begin{equation}
J\frac{d\omega_{m}}{dt}=T_{m}-T_{e}=T_{a}
\label{eqmotion}
\end{equation}

In power system network, the power ratings of the generators and corresponding moment of inertia values varies. Hence, it is more convenient to use inertia constant, $H$ whose unit is seconds and varies between 2 and 9. Inertia constant is defined as the ratio of kinetic energy stored in the inertia to the power rating of the generator as in Eq.\ref{inertiaconstant} where $\omega_{0m}$ denotes the rated angular velocity of generator in rad/s and $S_{base}$ is the rated apparent power in VA. 
\begin{equation}
H=\frac{{\frac{1}{2}}J\omega_{0m}^{2}}{S_{base}}
\label{inertiaconstant}
\end{equation}

Substituting Eq.\ref{inertiaconstant} into Eq.\ref{eqmotion} and replacing units to per-unit quantities yield the relation of frequency with power and inertia constant as in Eq.\ref{eqmotion5}

\begin{equation}
J=\frac{2H}{\omega_{0m}^{2}}{S_{base}}
\label{inertiaconstant2}
\end{equation}

\begin{equation}
\frac{2H}{\omega_{0m}^{2}}{S_{base}}\frac{d\omega_{m}}{dt}=T_{m}-T_{e}
\label{eqmotion2}
\end{equation}

\begin{equation}
\frac{2H}{\omega_{0m}^{2}}{S_{base}\omega_{m}}\frac{d\omega_{m}}{dt}=P_{m}-P_{e}
\label{eqmotion3}
\end{equation}

\begin{equation}
2H\frac{\omega_{m}}{\omega_{0m}}\frac{d(\omega_{m}/\omega_{0m})}{dt}=\frac{P_{m}-P_{e}}{S_{base}}
\label{eqmotion4}
\end{equation}

\begin{equation}
2H\overline{\omega_{m}}\frac{d\overline{\omega_{m}}}{dt}=\overline{P_{m}}-\overline{P_{e}}
\label{eqmotion5}
\end{equation}

\subsection{Frequency in Power Systems}
The frequency in a power system changes according to the swing equation. The equation basically investigates the relation between mechanical and electromechanical powers and the rate of change of angular speed of a generator. Therefore, the speed of an generator remains constant if the mechanical and electromechanical powers are equal.\\
The electricity grid can also be thought as a single generator whose inertia constant is aggravated from each generator in the network. In this case, average frequency in the network can be found as in Equation \ref{systemswing}. 

\begin{equation}
\label{systemswing}
2H_{sys}\overline{f_{sys}}\frac{d\overline{f_{sys}}}{dt}=\overline{P_{m}}-\overline{P_{e}}
\end{equation}

In the Equation \ref{systemswing}, $P_{m}$ is the aggravated mechanical input of the generators meanwhile $P_{e}$ is the aggravated electromechanical output. In other words, the system frequency depends on the balance between generation and consumption. Note that generation means the input mechanical power of the generators. \\
The behaviour of the frequency in electric grid is given in Figure \ref{frequencyingrid}. As it can be seen from the water level in a container analogy, the frequency of the system is dependent on the in-flow and the out-flow. Therefore, in the electricity grid, frequency increases as the aggravated input power is higher than the aggravated output power. Note that, the direction of the frequency is dictated by this balance. Having a constant 49.8Hz frequency does not mean that consumption is higher than generation.

\begin{figure}[h!]
	\centering
	\includegraphics[width=.95\linewidth]{frequencypool.png}
	\caption{Frequency behaviour in electric grid with the water level in a container analogy \cite{Eto2010}}
	\label{frequencyingrid}
\end{figure} 

Having a constant frequency is one of the most important responsibilities of a system operator. In order to have a constant frequency, supply is being adjusted according to the demand continuously. By doing so, the system frequency varies between a band-gap. The variation depends on the disturbances which are generally a sudden generation outage or instant load connection. The size of the disturbance determines the severity of the frequency change and there are three main mechanisms to arrest the frequency changes in the system. 

\subsubsection{Primary Frequency Controllers}

