% CHAPTER 1
\chapter{VALIDATION IN TEST CASE}
\label{chp:5}
\section{P.M.Anderson 9 Bus Test Case}
\subsection{System Properties}
In order to understand frequency dynamics better, P.M. Anderson test case has been used in the study. The single line diagram of the system is given in Fig. \ref{ieee_9_bus}. The test case consists of three generators and three loads. Generators in the system are connected to 230 kV high voltage network with transformers.\par
\begin{figure}[h!]
	\centering
	\includegraphics[width=.85\linewidth]{ieee_9_bus.png}
	\caption{P.M.Anderson Test Case}
	\label{ieee_9_bus}
\end{figure}
The biggest generator in the system is a hydro power plant with a power rating of 247.5 MVA. The remaining ones are steam generators. The power ratings of the generators are given in Table \ref{generatorproperties}.\par
\begin{table}[h!]
	\centering
	\begin{tabular}{ccc}
		\hline
		\textbf{Generators} & \textbf{Power Rating (MVA)} & \textbf{Plant Type} \\ \hline
		Gen 1               & 247.5                       & Hydro				\\
		Gen 2               & 192                         & Steam               \\
		Gen 3               & 128                         & Steam               \\ \hline
	\end{tabular}
	\caption{Generator Properties of Test System}
	\label{generatorproperties}
\end{table}
The loads in the system are connected directly to the high voltage network. The active and reactive power ratings of the loads are listed in Table \ref{loadproperties}.
\begin{table}[h!]
	\centering
	\begin{tabular}{ccc}
		\hline
		\textbf{Generators} & \textbf{Active Power (MW)}  & \textbf{Reactive Power (MVAr)} \\ \hline
		Load A              & 125                      	  & 50				 \\
		Load B              & 90                          & 30                \\
		Load C              & 100                         & 35                \\ \hline
	\end{tabular}
	\caption{Load Properties of Test System}
	\label{loadproperties}
\end{table}
\subsection{Load Flow Analysis for Base Case}
Successful grid operation requires a load flow analysis in order to ensure that bus voltages are inside the allowed band and power flows are below the power carrying capabilities of the lines. Load flow results are given in Table \ref{loadflow_case1}.
\begin{table}[h!]
	\centering
	\begin{tabular}{cclccccc}
		\hline
		Bus \# & Bus Type & \multicolumn{1}{c}{Voltage} & Angle & Pg    & Qg     & Pl  & Ql \\ \hline
		1      & SL       & \multicolumn{1}{c}{1.04}    & 0     & 71.65 & 27.05  & 0   & 0  \\
		2      & PV       & \multicolumn{1}{c}{1.025}   & 9.28  & 163   & 6.65   & 0   & 0  \\
		3      & PV       & \multicolumn{1}{c}{1.025}   & 4.66  & 85    & -10.86 & 0   & 0  \\
		4      & PQ       & 1.0258                      & -2.22 & 0     & 0      & 0   & 0  \\
		5      & PQ       & 0.9956                      & -3.99 & 0     & 0      & 125 & 50 \\
		6      & PQ       & 1.0126                      & -3.69 & 0     & 0      & 90  & 30 \\
		7      & PQ       & 1.0258                      & 3.72  & 0     & 0      & 0   & 0  \\
		8      & PQ       & 1.0159                      & 0.73  & 0     & 0      & 100 & 35 \\
		9      & PQ       & 1.0323                      & 1.97  & 0     & 0      & 0   & 0  \\ \hline
	\end{tabular}
	\caption{Load Flow Results in Base Case}
	\label{loadflow_case1}
\end{table}
\subsection{Base Case Frequency Response for Additional Load Connection}
It is obvious that power system networks experience high RoCoF when either high amount of generation trips or high amount of load connects to system. These two main event can be used in the simulation to create frequency disturbances. Since the simulation in Simulink environment slows down with the increasing amount of generators, the disturbances are created with load connections.\par
\begin{table}[]
	\centering
	\begin{tabular}{ll}
		\hline
		Total System Load                      & 315 MW    \\
		Generator Droop Settings               & 5\%       \\
		Stored Kinetic Energy at Nominal Speed & 3.305 GWs \\
		Gen 1 Inertia Constant                 & 9.5515 s  \\
		Gen 2 Inertia Constant                 & 3.9216 s  \\
		Gen 3 Inertia Constant                 & 2.7665 s  \\ \hline
	\end{tabular}
	\caption{System Dynamical Properties}
	\label{systemdynamicaldata}
\end{table}
System dynamical properties are listed in Table \ref{systemdynamicaldata}. Power generation references are determined based on the load flow of powergui toolbox. Machine initialization toolbox is also used to initiate the state of generators in the system. However, the system does not start with the steady state. Still, system goes to steady state within a few seconds. Frequency of the network is disturbed with a load connection in the t=10 seconds in order to observe the frequency stability of the system. For 10\% load connection, a load of 31.5 MW is connected to system from Bus 6. Location of the additional load is depicted in Fig. \ref{ieee_9_bus_load}.\par
\begin{figure}[h!]
	\centering
	\includegraphics[width=.85\linewidth]{ieee_9_bus_load_position.png}
	\caption{Location of the Additional Load}
	\label{ieee_9_bus_load}
\end{figure}
According to the 10\% load connection to system, generator frequencies are shown in Fig. \ref{genfreqcase1}. Frequency of generator 1 is the most smooth one due to its huge inertia constant. Meanwhile, the generator 2 and generator 3 follow the frequency of generator 1 even though they oscillate with each other. \par
\begin{figure}[h!]
	\centering
	\includegraphics[width=.85\linewidth]{Case1_Generator_Freq.png}
	\caption{Generator Frequencies for 10\% Load Connection}
	\label{genfreqcase1}
\end{figure}
In the system, frequency of Bus 1 can be assumed as constant throughout the network since the system is small enough to assume a single frequency. This assumption can be verified by comparing the frequencies in Buses 1, 5 and 6. Fig. \ref{genfreqcase1_loadgen} shows the frequency of the generator 1 frequency as well as the load frequencies captured with Simulink PLL block. The only difference is the instant following the load connection. The sharp frequency decline delays the PLL loop to capture the frequnecy. 
\begin{figure}[h!]
	\centering
	\includegraphics[width=.85\linewidth]{Case1_Load_Gen_Freq.png}
	\caption{Frequencies in Generator 1, Load A and Load B}
	\label{genfreqcase1_loadgen}
\end{figure}
\section{Modified Case}
In this case, the P.M. Anderson test case is modified such that a wind farm consists of 20 wind turbine is connected to network. Wind farm is connected to Bus 5. Modified system is depicted in the Fig. \ref{ieee_9_bus_case2}. In this case, generator 2 and 3 are still assigned to same power values meanwhile generator 1 decreases its generation. 
\begin{figure}[h!]
	\centering
	\includegraphics[width=.85\linewidth]{ieee_9_bus_modified.png}
	\caption{Modified System Single Line Diagram}
	\label{ieee_9_bus_case2}
\end{figure}
\subsection{Load Flow Analysis for Modified Case}
Load flow analysis for modified case is listed in Table \ref{loadflow_case2}. The power injected from Bus 1 is decreased as expected. This can also be seen from the phase angle between 1 and 4. Phase angle difference between these buses decreased from $2.22^{\circ}$ to $1.18^{\circ}$.
\begin{table}[h!]
	\centering
	\begin{tabular}{cclccccc}
		\hline
		Bus \# & Bus Type & \multicolumn{1}{c}{Voltage} & Angle & Pg    & Qg     & Pl  & Ql \\ \hline
		1      & SL       & \multicolumn{1}{c}{1.04}    & 0     & 38.06 & 25.07  & 0   & 0  \\
		2      & PV       & \multicolumn{1}{c}{1.025}   & 11.33 & 163   & 6.65   & 0   & 0  \\
		3      & PV       & \multicolumn{1}{c}{1.025}   & 6.32  & 85    & -10.86 & 0   & 0  \\
		4      & PQ       & 1.0263                      & -1.18 & 0     & 0      & 0   & 0  \\
		5      & PQ       & 0.9995                      & -1.54 & 0     & 0      & 125 & 50 \\
		6      & PQ       & 1.0128                      & -2.43 & 0     & 0      & 90  & 30 \\
		7      & PQ       & 1.0266                      & 5.77  & 0     & 0      & 0   & 0  \\
		8      & PQ       & 1.0164                      & 2.62  & 0     & 0      & 100 & 35 \\
		9      & PQ       & 1.0326                      & 3.62  & 0     & 0      & 0   & 0  \\ \hline
	\end{tabular}
	\caption{Load Flow Results for Modified Case}
	\label{loadflow_case2}
\end{table}
\subsection{Modified Case Frequency Response for Additional Load Connection}
The modified base is very similar to the Base Case except for a wind farm located in Bus 5. The renewable energy system in this case can be considered as a negative load. Therefore, base case with decreased load is under discussion in this subsection. The same amount of load is taken into operation at Bus 6 and the frequency of the system is shown in Fig. \ref{Case1_2_freq}.\\
\begin{figure}[h!]
	\centering
	\includegraphics[width=.85\linewidth]{Case1_2.png}
	\caption{Comparison of Base Case and Modified Case Frequencies}
	\label{Case1_2_freq}
\end{figure}
Almost the same frequency response is observed in the system. The reason is that both systems have the same amount of stored kinetic energy. Another reason is the underutilization of the power system network. This can also be observed in the rate of change of frequencies in Fig. \ref{Case1_2_rocof}. Almost the same RoCoF values are observed in the system. This concludes that renewable energy penetration does not change frequency response of the system if the only change in the system is the inclusion of renewable energy system. Note that the renewable energy systems are intermittent energy sources. However, in this study,the source of the renewable energy system is assumed as constant. Therefore, the reason of frequency disturbance is load connection rather than the change in active power output of renewable systems.
\begin{figure}[h!]
	\centering
	\includegraphics[width=.85\linewidth]{Case1_2_rocof.png}
	\caption{Comparison of Base Case and Modified Case Frequencies}
	\label{Case1_2_rocof}
\end{figure}
\section{Decommissioned Case}
As seen in the Modified Case, the frequency response of the system does not change with renewable energy inclusion. However, it is inevitable that renewable energy systems will replace the conventional units in future. In this case, the smallest generator, generator 3, will be decommissioned. The decommissioned case diagram is shown in Fig. \ref{decommissioned_case}.\par
\begin{figure}[h!]
	\centering
	\includegraphics[width=.85\linewidth]{ieee_9_bus_decommissioned.png}
	\caption{Comparison of Base Case and Modified Case Frequencies}
	\label{decommissioned_case}
\end{figure}
Since the generator 3 is out of service, the stored kinetic energy is decreased in the system. Decommissioned system dynamical properties are updated and given in Table \ref{systemdynamicaldatacase3}.
\begin{table}[]
	\centering
	\begin{tabular}{ll}
		\hline
		Total System Load                      & 315 MW    \\
		Generator Droop Settings               & 5\%       \\
		Stored Kinetic Energy at Nominal Speed & 3.004 GWs \\
		Gen 1 Inertia Constant                 & 9.5515 s  \\
		Gen 2 Inertia Constant                 & 3.9216 s  \\
		\hline
	\end{tabular}
	\caption{System Dynamical Properties}
	\label{systemdynamicaldatacase3}
\end{table}
\subsection{Load Flow Analysis for Decommissioned Case}
Since the generator 3 is out of service, generator 1 loading will be increased. Load flow analysis for decommissioned case is given in Table \ref{loadflow_case3}.
\begin{table}[h!]
	\centering
	\begin{tabular}{cclccccc}
		\hline
		Bus \# & Bus Type & \multicolumn{1}{c}{Voltage} & Angle & Pg    & Qg     & Pl  & Ql \\ \hline
		1      & SL       & \multicolumn{1}{c}{1.04}    & 0     & 121.76& 16.26  & 0   & 0  \\
		2      & PV       & \multicolumn{1}{c}{1.025}   & 4.18  & 163   & 0.65   & 0   & 0  \\
		4      & PQ       & 1.0332                      & -3.74 & 0     & 0      & 0   & 0  \\
		5      & PQ       & 1.0083                      & -5.63 & 0     & 0      & 125 & 50 \\
		6      & PQ       & 1.0224                      & -7.65 & 0     & 0      & 90  & 30 \\
		7      & PQ       & 1.0294                      & -1.36 & 0     & 0      & 0   & 0  \\
		8      & PQ       & 1.0207                      & -5.82 & 0     & 0      & 100 & 35 \\
 \hline
	\end{tabular}
	\caption{Load Flow Results for Decommissioned Case}
	\label{loadflow_case3}
\end{table}
\subsection{Decommissioned Case Frequency Response for Additional Load Connection}
Same amount of additional load is taken into operation from Bus 6. System frequency response is observed and compared to Base Case and Modified Case in Fig. \ref{Case1_2_3_freq}. As soon from the figure, the frequency nadir decreased from 49.77Hz to 49.65Hz. This is due to the decrease in the stored kinetic energy in the system. Due to the decommission of generator 3, the frequency decreases steeper following to load connection.This can also be observed RoCoF comparison given in Fig. \ref{Case1_2_3_rocof}. 
\begin{figure}[h!]
	\centering
	\includegraphics[width=.85\linewidth]{Case1_2_3_freq.png}
	\caption{Comparison of Base Case,Modified Case and Decommissioned Case Frequency Responses}
	\label{Case1_2_3_freq}
\end{figure}
\begin{figure}[h!]
	\centering
	\includegraphics[width=.85\linewidth]{Case1_2_3_rocof.png}
	\caption{Comparison of Base Case,Modified Case and Decommissioned Case RoCoFs}
	\label{Case1_2_3_rocof}
\end{figure}
