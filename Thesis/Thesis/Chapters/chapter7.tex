% CHAPTER 1
\chapter{CONCLUSION}
\label{chp:7}
This thesis study highlights the problem of reduction in effective grid inertia that is caused by increasing renewable energy penetration. Since the renewable energy systems that are connected to grid with a power electronics interface decouples rotation speed and grid frequency, the inertia of these units do not contribute the frequency stability of the grid. However, it is shown that the reduction in the grid inertia can be compensated with wind turbines with FSPC. It is even possible to improve the frequency stability of the grid with the implementation of the inertial support mechanisms.\par
The inertial support implementations either frequency dependent or not can be provided with the utilization of the additional energy in the renewable energy systems. The increase in the active power is found to be limited by the converter capability meanwhile the available kinetic energy restricts the support duration. This thesis study demonstrates the relation between the available increase in the active power and wind speed. It is observed that the increase in the active power is limited as 0.1pu in high wind speeds. However, the additional active power up to 0.45pu can be injected to grid when the wind speed is lower than 10m/s. The shortest support interval is slightly above 1 second and observed in the wind speed 3m/s. Therefore, it is concluded that the kinetic energy affects the support duration but the increase in active power is limited by wind speeds. \par
For maximum RoCoF of 0.25Hz/s, the studied wind turbine can emulate the inertial constant up to 32seconds. Nonetheless, the inertia constant H=10s is applicable to whole wind speed range. By utilizing the synthetic inertia implementation H=10s, the effective kinetic energy existing in the grid inertia can be increased by 8\% in the Turkish electricity grid. Besides, the reduction in the effective grid inertia constant can be compensated with the synthetic inertia implementation of 10 seconds. Consequently, the share of the renewable energy can be increased without deteriorating the effective grid inertia.\par
The additional active power in the low and medium wind speed ranges are provided with the kinetic energy extraction from the turbine inertia. However, the additional power can be released from the turbine capacity by decreasing the pitch angle. In both cases, the active power can be increased within 20ms by considering the electrical time constant. However, the detection of the grid RoCoF and calculation of the required power and current references bring additional delays. Therefore, the real implementation on a wind turbine might be less effective compared to simulation case in which only the electrical time constant is under consideration.\par
This study also evaluates the economical perspective of the inertial support. The payment to inertia support provision is important to convince the energy provider for implementing the grid supporting applications. It is observed that the payment by considering the additional power is insignificant compared to daily average profit of the owner. In contrary, the significant amount of additional payment can be obtained as long as incentives are paid to renewable energy units for grid supporting applications. In this case, the payment is regardless from the additional energy but for the availability of the unit to the synthetic inertia implementation. In this way, the energy providers can be convinced to implement grid supporting applications that will increase the stability of the electricity grid.