% CHAPTER 1
\chapter{EVALUATION OF FAST INERTIAL RESPONSE AND SYNTHETIC INERTIA IMPLEMENTATION}
\label{chp:6}
The installed capacity of power systems grows dramatically each passing day. Old fashion electricity grid is composed of conventional synchronous generators whose huge percentage is supplied from fossil fuels. Due to decrease the percentage of fossil fuel based generation, renewable energy systems which do not have environmental results are preferred and supported. Huge amount of renewable energy systems are connected to electricity grid during these two decades. Now, the debates try to answer whether electricity grid with only renewable energy systems are possible or not. \par
However, increase in the share of renewable energy in the installed capacity brought operational problems as well. As the total installed capacity grows, the rotational mass in the electricity grid slightly rises. This is due to the fact that installed PV systems do not have rotational bodies at all or wind turbines with full-scale power electronics does not effectively contribute the grid aggregated inertia. Hence, the power system with high renewable energy penetration is exposed to high RoCoF for the frequency disturbances. This implies that system will encounter unacceptable RoCoF values in frequency disturbances as long as the renewable penetration continues. Therefore, the upcoming future power system will require auxiliary services such as synthetic or virtual inertial support from all generation technologies that includes power electronics. \par
Renewable energy systems produces power according to the type of its power source. The input power is constant for an instant since the source (solar radiation, wind speed etc.) is constant. Therefore, the operation of renewable systems is different from the conventional power plants whose input power can be increased steadily by the operator. Nonetheless, an additional energy source is required in the renewable systems that is stored kinetic energy in the wind turbine case. Therefore, the wind turbines are able to increase their active output power by extracting the kinetic energy stored in the turbine equivalent inertia. However, the amount of active power increase can be pre-defined values by fast provision as in the case of Chapter \ref{chp:4} or it can be proportional to RoCoF as in the case of Chapter \ref{chp:5}.
\section{Fast Inertial Support}
Wind energy systems with full-scale power electronics can adjust its active power by controlling its output torque. Therefore, the active power can be quickly raised by extracting the energy in turbine inertia. However, the maximum value of the active power depends on the operating condition. Since the generator active power is also dependent on its speed, turbine power cannot be increased to rated power in low wind speeds. In high wind speeds, the active power before the support is already close to its rated value. Therefore, the increase in the active power is much more limited in high wind scenarios. In this study, fast inertial response in the medium wind speed is found out to have much more potential the other scenarios. Besides, wind turbines can contribute better in low wind scenarios than that of high wind speed.\par
It is to note that the frequency disturbance occurs due to the unbalance between input mechanical power and output powers of the generators. Hence, the additional amount of active power that is provided from renewable energy sources in such instants is favourable. This is why the increase in the active power is much more important than the final active power amount. Therefore, knowledge of active power limits reveals the potential of variable speed wind turbines in order to contribute the frequency stability of the power systems.\par
Fast inertial response can be provided in different amount of time durations. Since the larger amount of support might result in higher speed deviations, the support time might be decreased. In contrary, higher support durations can be achieved with lower amount of fast inertial response.\par
Fast inertial response in this study is not a direct function of the RoCoF as in the case of synthetic inertia implementation. However, an RoCoF indexing can be constructed for fast inertial support provision. The indexing scheme requires a RoCoF threshold and different RoCoF intervals. The highest RoCoF interval corresponds to the most severe frequency disturbances case and requires the highest amount of fast inertial support release with highest available support time. The higher energy extraction might even result in the stall of the turbine. Nonetheless, critical instant following the disturbance is much more important than the turbine speed recovery. Meanwhile, the lowest RoCoF interval would be assigned to lowest inertial support release with time duration in order not to result higher speed deviation. 
\section{Synthetic Inertia Implementation}
Even though wind turbine power can be changed as desired inside an allowable power range, the fastest release of the power is not a solution for the power system network. Although huge amount of power is released in the disturbances, the restoration of that energy to the turbine causes further problems to grid. Therefore, the increase amount should be in coordination with the grid frequency behaviour. This is ensured with the synthetic inertia implementation in the Chapter \ref{chp:5}. By adjusting active power according to the RoCoF of the grid, the active power is increased or decreased depending on the grid status.\par
The P.M. Anderson test case constructed in Matlab-Simulink environment is first tested with a load connection to see the frequency response of the system. The constructed system frequency falls at least for 2 seconds until the conventional generators increase their active power output. Frequency of the three generators are observed for the disturbance. In the system, generator 1 has the biggest inertia constant. Therefore, when the frequency of the system falls, there are rotor swings between generator 2 and 3 that follows the frequency of the generator 1. In the modified case, a wind farm with 20 wind turbines is connected to system in Bus 5. The wind farm connection does not change the generator 2 and 3 reference values but the generator 1 production. When the modified system is subjected to the same frequency disturbance, the system frequency response is not affected at all. The reason is that the test system has exactly the same kinetic energy or the effective inertia. Inertia existing in the wind farm does not contribute the system frequency stability. Meanwhile the wind farm production can be considered as negative load whose power is not dependent on the frequency of the system.\par
The common concern in the frequency stability reports is not the renewable energy penetration itself but the renewable energy penetration that replaces the conventional synchronous generators. System frequency stability is subjected to change especially renewable energy systems are preferred over the conventional generation units. Therefore, the system frequency response gets more critical when a unit is decommissioned in the economical dispatch. To investigate this issue, the generator 3 is taken off from the system. Therefore, kinetic energy in the system declined that increased the responsibility of the remaining generators in the system. The test system in the decommissioned case is exposed to lower RoCoF and lower post-disturbance frequency.  The reason in the lower RoCoF is the decreased rotational energy in the system i.e. lower inertia. Moreover, the lower post-disturbance is caused by the decreased primary reserve due to the generator 3 outage.\par
The most important feature of the synthetic inertia implementation is the RoCoF dependency of the support. The support in this method does increase and decrease the power value according to the pre-disturbance value. For instance, the power is above the pre-disturbance power value when the RoCoF is negative. In contrast, the lower amount of power injected to grid as soon as the RoCoF is positive. In this way,turbine recover its speed by injecting lower power to grid.\par
The synthetic inertia implementation is used for improving the transient behaviour of the frequency. In the literature variety of inertia constants are emulated in variable speed wind turbines. However, the emulated inertia constant are generally inside 2 and 10. Nonetheless, it is observed that the wind turbines have enormous capacity to increase their power. Therefore, in this study inertia constants more than 10s are also tested. However, further increasing inertia constants to be emulated would extract high amount of active power which increases the restoration time. Since the wind speed would not be constant for long time, increased inertia constants might cause problems.\par 
\section{Economical Motivations for Energy Providers}
As explained in Section \ref{section-price}, renewable energy systems in Turkey and most of the EU countries sell electricity with feed-in tariff. It basically means that all the generated energy will be bought for sure without trading inside the market. Even though the problems are arising with renewable energy systems, the energy providers would not be a volunteer for ancillary services unless the regulations impose sanctions or additional payments. Hence, the system operator should prepare more advantageous paying mechanisms in order to persuade energy providers to implement grid supporting methods. \par
The system operators start preparing new mechanisms for frequency regulations. One of the examples is the Firm Frequency Response (FFR) by National Grid. FFR is basically frequency support method that is activated with the frequency thresholds. As the frequency falls below a pre-determined value, the response is needed from energy providers (either from synchronous generator or energy storage units). These energy providers are taken into operation according to their tenders. The response is either non-dynamic (independent from the frequency shift) or dynamic (pre-determined percentage increase according to frequency). Moreover, the  support should be continuous for 30 seconds in primary response \cite{Smethurst2017}. This is why the response is provided by synchronous generators and energy storage systems. It is to note that this mechanism is not appropriate for the renewable energy systems whose output power is increased for shortest time intervals. Thus, the energy storage systems are encouraged for such applications. \par
Another frequency regulating mechanism is applied in USA according to the FERC 755 regulation. The frequency regulation includes different metrics for payment such as capacity, performance and mileage. Based on this regulation, energy price by high performance frequency regulation resources is increased over three times of the old price of the PJM which is a regional transmission company \cite{NECEnergySolutions2014}.\par
When the wind turbine is used for inertial support mechanisms, shortest time periods and smaller energy is provided. Therefore, the payment to energy provider for the additional amount of energy would not be beneficial.

\begin{table}[]
	\centering
	\begin{tabular}{ccc}
		\hline
		\multicolumn{3}{c}{\textbf{Base Case Profit}}                                                          \\
		Generated Energy                                                                      & 23    & MWh    \\
		Feed-In Tariff                                                                        & 73    & \$/MWh \\
		Daily Generation                                                                      & 1679  & \$     \\ \hline
		\multicolumn{3}{c}{\textbf{Profit by Additional Energy}}                                               \\
		\multicolumn{1}{l}{Energy From Support Period}                                        & 22.15 & kWh    \\
		\begin{tabular}[c]{@{}c@{}}Supported Energy Price\\ (3.4xFeed-In Tariff)\end{tabular} & 248.2 & \$/MWh \\
		\multicolumn{1}{l}{Additional Profit}                                                 & 5.5   & \$     \\ \hline
		\multicolumn{3}{c}{\textbf{Profit with Incentive}}                                                     \\
		\multicolumn{1}{l}{Generated Energy}                                                  & 23    & MWh    \\
		\multicolumn{1}{l}{Supported Feed-In Tariff}                                          & 79    & \$/MWh \\
		\multicolumn{1}{l}{Additional Profit}                                                 & 138   & \$     \\ \hline
	\end{tabular}
	\caption{Comparison of the Frequency Support Pricing Methods}
	\label{new_price}
\end{table}

\section{FUTURE WORK}
\begin{itemize}
	\item Fast inertial support implementation has huge potential to increase the active power. However, as soon as the support ends, the decrease in the active power resembles a second load connection to system. Therefore, the system is exposed to a secondary decrease in the frequency. This is why the amount of additional power should be adjusted according to the grid requirements. Hence, a support index can be developed to reshape the fast inertial support. 
	\item Fast inertial support in this study is activated according to the RoCoF threshold of 0.1Hz/s. However, the increase percentage is not a function of the grid parameters. Thus, an index might be constructed for the on-line employment of the fast inertial support. The index to be constructed will determine the increase percentage based on the pre-determined values. 
	\item The effects of the fast inertial support on the DC-link voltage might be better investigated with more realistic modelling. The effectiveness of the new modelling might be tested in the wind turbine emulators. 
	\item When the support is required in the high wind speeds, support power might hit the maximum allowed converter powers. The extra heat and mechanical stresses might also be tested in wind turbine emulators. 
\end{itemize}