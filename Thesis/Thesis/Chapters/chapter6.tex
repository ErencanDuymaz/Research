% CHAPTER 1
\chapter{EVALUATION OF FAST INERTIAL RESPONSE AND SYNTHETIC INERTIA IMPLEMENTATION}
\label{chp:6}
The installed capacity of power systems grows dramatically each passing day. Old fashion electricity grid is composed of conventional synchronous generators whose huge percentage is supplied from fossil fuels. Due to decrease the percentage of fossil fuel based generation, renewable energy systems which do not have environmental results are preferred and supported. Huge amount of renewable energy systems are connected to electricity grid during these two decades. Now, the debates try to answer whether electricity grid with only renewable energy systems are possible or not. \par
However, increase in the share of renewable energy in the installed capacity brought operational problems as well. As the total installed capacity grows, the rotational mass in the electricity grid slightly rises. This is due to the fact that installed PV systems do not have rotational bodies at all or wind turbines with full-scale power electronics does not effectively contribute the grid aggregated inertia. Hence, the power system with high renewable energy penetration is exposed to high RoCoF for the frequency disturbances. This implies that system will encounter unacceptable RoCoF values in frequency disturbances as long as the renewable penetration continues. Therefore, the upcoming future power system will require auxiliary services such as synthetic or virtual inertial support from all generation technologies that includes power electronics. \par
Renewable energy systems produces power according to the type of its power source. The input power is constant for an instant since the source (solar radiation, wind speed etc.) is constant. Therefore, the operation of renewable systems is different from the conventional power plants whose input power can be increased steadily by the operator. Nonetheless, an additional energy source is required in the renewable systems that is stored kinetic energy in the wind turbine case. Therefore, the wind turbines are able to increase their active output power by extracting the kinetic energy stored in the turbine equivalent inertia. However, the amount of active power increase can be pre-defined values by fast provision as in the case of Chapter \ref{chp:4} or it can be proportional to RoCoF as in the case of Chapter \ref{chp:5}.
\section{FAST INERTIAL RESPONSE}
Wind energy systems with full-scale power electronics can adjust its active power by controlling its output torque. Therefore, the active power can be quickly raised by extracting the energy in turbine inertia. However, the maximum value of the active power depends on the operating condition. Since the generator active power is also dependent on its speed, turbine power cannot be increased to rated power in low wind speeds. In high wind speeds, the active power before the support is already close to its rated value. Therefore, the increase in the active power is much more limited in high wind scenarios. In this study, fast inertial response in the medium wind speed is found out to have much more potential the other scenarios. Besides, wind turbines can contribute better in low wind scenarios than that of high wind speed.\par
It is to note that the frequency disturbance occurs due to the unbalance between input mechanical power and output powers of the generators. Hence, the additional amount of active power that is provided from renewable energy sources in such instants is favourable. This is why the increase in the active power is much more important than the final active power amount. Therefore, knowledge of active power limits reveals the potential of variable speed wind turbines in order to contribute the frequency stability of the power systems.\par
Fast inertial response can be provided in different amount of time durations. Since the larger amount of support might result in higher speed deviations, the support time might be decreased. In contrary, higher support durations can be achieved with lower amount of fast inertial response.\par
Fast inertial response in this study is not a direct function of the RoCoF as in the case of synthetic inertia implementation. However, an RoCoF indexing can be constructed for fast inertial support provision. The indexing scheme requires a RoCoF threshold and different RoCoF intervals. The highest RoCoF interval corresponds to the most severe frequency disturbances case and requires the highest amount of fast inertial support release with highest available support time. The higher energy extraction might even result in the stall of the turbine. Nonetheless, critical instant following the disturbance is much more important than the turbine speed recovery. Meanwhile, the lowest RoCoF interval would be assigned to lowest inertial support release with time duration in order not to result higher speed deviation. 
\section{SYNTHETIC INERTIA IMPLEMENTATION}

\section{FUTURE WORK}