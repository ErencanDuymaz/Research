% CHAPTER 1
\chapter{EVALUATION OF FAST INERTIAL RESPONSE AND SYNTHETIC INERTIA IMPLEMENTATION}
\label{chp:6}
Increase in the share of renewable energy in the installed capacity brought operational problems. Due to the fact that PV systems do not have rotational mass at all or wind turbines with full-scale or partial scale power electronics do not effectively contribute the grid aggregated inertia, the power system with increasing renewable energy penetration will be exposed to high RoCoF for the frequency disturbances. This implies that system will encounter unacceptable RoCoF values (around the RoCoF protection settings of the generating units) in the normal operation as long as the renewable penetration continues. Therefore, the upcoming future power system will require auxiliary services such as synthetic or virtual inertial support from all generation technologies that includes power electronics. \par
Renewable energy systems produces power according to the type of its power source. The input power is constant for an instant since the source (solar radiation, wind speed etc.) is constant. Therefore, the operation of renewable systems is different from the conventional power plants whose input power can be increased steadily by the operator. Nonetheless, an additional energy source is required in the renewable systems in order to increase active power as desired. For this purpose, kinetic energy in wind turbine blades and generator can be utilized. Therefore, the wind turbines are able to increase their active output power by extracting the kinetic energy stored in the turbine equivalent inertia. However, the amount of active power increase can be either non-dynamic as in the case of Chapter \ref{chp:4} or dynamic (proportional to RoCoF) as in the case of Chapter \ref{chp:5}.
\section{Fast Inertial Support}
Wind energy systems with full-scale power electronics can adjust its active power by controlling its output torque. Therefore, the active power can be quickly raised by extracting the stored energy in turbine inertia. However, the maximum value of the active power depends on the operating conditions. Since the generator active power is also dependent on its speed, turbine power cannot be increased up to rated power but to half of the rated power in wind speeds lower than 5m/s. In wind speeds above 10m/s, the active power before the support is already close to its rated value. Therefore, the increase in the active power is limited as 10\% (as long as converter rating has the capacity) in high wind scenarios. \par
In this study, the wind turbine can increase its active output power by 0.45pu in the wind speeds between 5m/s and 8m/s for fast inertial response. The highest active power release is found in the wind speed 6.5m/s as 0.49pu. Besides, wind turbines can contribute better in low wind scenarios than that of high wind speed for short time intervals.\par
It should be noted that the frequency disturbance occurs due to the unbalance between input mechanical power and output powers of the generators. Hence, the additional amount of active power that is provided from renewable energy sources in such instants is highly favourable. This is why the increase in the active power is much more important than the final active power amount. Therefore, knowledge of active power limits reveals the potential of variable speed wind turbines in order to contribute the frequency stability of the power systems.\par
Fast inertial response can be provided in different amount of time durations up to 30 seconds. Since the larger amount of support might result in higher speed deviations, the support time might be decreased. In contrary, higher support durations can be achieved with lower amount of fast inertial response.\par
Fast inertial response in this study is not a direct function of the RoCoF. In other words, the support power is independent from RoCoF or frequency deviation. However, an RoCoF indexing can be developed to obtain different support power values. In this case, the indexing scheme requires a RoCoF threshold and different RoCoF intervals. The highest RoCoF interval corresponds to the most severe frequency disturbances case and requires the highest amount of fast inertial support release with highest available support time. Meanwhile, the lowest RoCoF interval would be assigned to lowest inertial support release with time duration in order not to result higher speed deviation. It should be noted that the higher energy extraction might even result in the stall of the turbine. Nonetheless, critical instant following the disturbance is much more important than the turbine speed recovery. Hence, grid operators might choose to extract the available active power in the expense of turbine stall according to the optimized decision.
\section{Synthetic Inertia Implementation}
Even though wind turbine power can be changed as desired inside an allowable power range, the fastest release of the power independent of the frequency is not a favourable solution especially for weak power grids. Although huge amount of power is released in the disturbances, the restoration of that energy to the turbine causes further problems to electricity grid. Therefore, the increase amount should be in coordination with the grid frequency behaviour. This is why dynamic frequency response is obtained with the synthetic inertia implementation in the Chapter \ref{chp:5}. By adjusting active power according to the RoCoF of the grid, the active power is increased or decreased depending on the grid status.\par
In order to observe the effects of synthetic inertia implementation, a dynamical 9-bus test system is constructed in Matlab-Simulink environment. The test system is composed of the conventional generators and subjected to a frequency disturbance with a load connection. In order to see the effects of the renewable energy penetration, a wind farm with 20 turbines is connected to system. When the modified system is exposed to the same frequency disturbance, almost the same frequency disturbance is observed in the system. The reason is that the test system has exactly the same kinetic energy or the effective inertia. Inertia existing in the wind farm does not contribute the system frequency stability. Meanwhile the wind farm production can be considered as negative load whose power is not dependent on the frequency of the system.\par
The common concern in the frequency stability reports is not the renewable energy penetration itself but the case where the renewable energy systems replaces with the conventional synchronous generators. System frequency stability is subjected to change especially renewable energy systems are preferred over the conventional generation units. Therefore, the system frequency response gets more critical when a unit is decommissioned in the economical dispatch. To investigate this issue, the generator 3 is taken off from the system. Therefore, kinetic energy in the system declined that increased the responsibility of the remaining generators in the system. The test system in the decommissioned case is exposed to lower RoCoF and lower post-disturbance frequency.  The reason in the lower RoCoF is the decreased rotational energy in the system i.e. lower inertia. Moreover, the lower post-disturbance frequency is caused by the decreased primary reserve due to the generator 3 outage.\par
The most important feature of the synthetic inertia implementation is the RoCoF dependency. The inertial support in this method does decrease the active power as soon as RoCoF turns positive meaning that turbine speed recovery can be started. Thus, the speed recovery of the synthetic inertia method begins with the frequency nadir is reached.\par
The synthetic inertia implementation is used for improving the transient behaviour of the frequency. In the literature variety of inertia constants are emulated in variable speed wind turbines. However, the emulated inertia constant are generally inside 2 and 10. Nonetheless, it is observed that the wind turbines have enormous capacity to increase their power. Therefore, in this study inertia constants more than 10s are also tested. However, further increasing inertia constants to be emulated would extract high amount of active power which increases the restoration time. Since the wind speed would not be constant for long time, increased inertia constants might cause problems.\par 
\section{Economical Motivations for Energy Providers}
As explained in Section \ref{section-price}, renewable energy systems in Turkey and most of the EU countries sell electricity with feed-in tariff. It basically means that all the generated energy will be bought for sure without trading inside the market. Even though the problems are arising with renewable energy systems, the energy providers would not be a volunteer for ancillary services unless the regulations impose sanctions or additional payment is provided. Hence, the system operator should prepare more advantageous paying mechanisms in order to persuade energy providers to implement grid supporting methods. \par
The system operators has already started preparing new frequency regulations. One of the examples is the Firm Frequency Response (FFR) by National Grid. FFR is basically frequency support method that is activated with the frequency thresholds. As the frequency falls below a pre-determined value, the response is needed from energy providers (either from synchronous generator or energy storage units). These energy providers are taken into operation according to tenders. The response is either non-dynamic (independent from the frequency shift) or dynamic (pre-determined percentage increase according to frequency). Moreover, the support power should be sustained up to 30 seconds dynamic response and up to 30 minutes for non-dynamic response for the primary \cite{Smethurst2017}. This is why the response is provided by synchronous generators and energy storage systems whose active power values can be adjusted as desired. It is to note that this mechanism is not appropriate for the renewable energy systems whose output power is increased for a short time intervals. \par
Another frequency regulating mechanism is applied in USA according to the FERC 755 regulation. The frequency regulation includes different metrics for payment such as capacity, performance and mileage. Based on this regulation, energy price for the high performance frequency regulation resources is increased over three times of the old price of the PJM which is a regional transmission company \cite{NECEnergySolutions2014}. Since one of the metrics effecting the payment is performance, the regulation is advantageous for energy storage systems that can adjust its active power quickly thanks to their power electronics interface.\par
When the wind turbine is used for inertial support mechanisms, small energy is provided for short time period. Even though the turbine injects significant amount of power to grid, the additional amount of energy is negligible compared to energy produced in this time interval. Therefore, the payment to energy provider for the additional amount of energy would not be beneficial. Table \ref{new_price} shows the average energy generation according to the measurements from site as well as the hypothetical profits from inertial support based on either additional energy or incentive. It is obvious that the wind turbine in this study earns average 1679\$ each day. If the additional active power is sold with a 248.2\$/MWh price (3.4XFeed-In Tariff), 5.5\$ additional profit is yielded. This corresponds to 0.33\% of the daily profit. and it is insignificant to energy provider. \par
\begin{table}[h]
	\centering
	\begin{tabular}{ccc}
		\hline
		\multicolumn{3}{c}{\textbf{Base Case Profit}}                                                          \\
		Generated Energy                                                                      & 23    & MWh    \\
		Feed-In Tariff                                                                        & 73    & \$/MWh \\
		Daily Generation                                                                      & 1679  & \$     \\ \hline
		\multicolumn{3}{c}{\textbf{Profit by Additional Energy}}                                               \\
		\multicolumn{1}{l}{Energy From Support Period}                                        & 22.15 & kWh    \\
		\begin{tabular}[c]{@{}c@{}}Supported Energy Price\\ (3.4xFeed-In Tariff)\end{tabular} & 248.2 & \$/MWh \\
		\multicolumn{1}{l}{Additional Profit}                                                 & 5.5   & \$     \\ \hline
		\multicolumn{3}{c}{\textbf{Profit with Incentive}}                                                     \\
		\multicolumn{1}{l}{Generated Energy}                                                  & 23    & MWh    \\
		\multicolumn{1}{l}{Supported Feed-In Tariff}                                          & 79    & \$/MWh \\
		\multicolumn{1}{l}{Additional Profit}                                                 & 138   & \$     \\ \hline
	\end{tabular}
	\caption{Comparison of the Frequency Support Pricing Methods}
	\label{new_price}
\end{table}
A better payment mechanism for inertial support can be constructed with incentives. Note that renewable energy systems earns additional payment with local content bonus. A similar incentive can be assigned to wind turbines for grid supporting mechanism. If the inertial support implementation is paid with the same amount with the lowest local content bonus (0.6cent/kWh for local turbine tower), 138\$ additional profit is obtained. In this case, the energy provider can increase the average income by 8.2\%. Therefore, assigning incentives for the grid supporting services might be attractive for the energy providers. Moreover, system operators of the weak power grids might lean towards the inertial support by wind turbine operators even at the expense of the additional incentives to the energy providers.
\section{Future Work}
The following issues can be further studied in detail:
\begin{itemize}
	\item Fast inertial support implementation has huge potential to increase the active power. However, as soon as the support ends, the decrease in the active power resembles a second load connection to system especially in the weak power systems. Therefore, the system is exposed to a secondary decrease in the frequency. This is why the amount of additional power should be adjusted according to the grid frequency. Hence, a support index can be developed to reshape the fast inertial support. 
	\item The increase percentage of fast inertial support is not a function of the grid parameters. The index to be constructed should also determine the increase percentage based on the pre-determined values. 
	\item The effects of the fast inertial support on the DC-link voltage might be better investigated with more realistic modelling. The effectiveness of the new modelling might be tested in the wind turbine emulators. 
	\item When the support is required in the high wind speeds, support power might hit the maximum allowed converter powers. The extra heat and mechanical stresses might also be tested in wind turbine emulators. 
	\item Storage technologies would increase the support time and amount. Economical motivations can be united with storage technologies.
\end{itemize}