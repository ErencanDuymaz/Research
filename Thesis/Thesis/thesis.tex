\documentclass[chaparabic,ee,ms,12pt,oneandhalf]{metu}
\usepackage{appendix}
\usepackage{longtable}
\usepackage[pdftex]{hyperref}
\usepackage[all]{hypcap}
\usepackage{todonotes}
\usepackage{graphicx}
%\usepackage{caption}
\usepackage{subcaption}
\graphicspath{ {./images/} }
\usepackage{rotating}
\usepackage{xy} 
\usepackage{booktabs}
\usepackage{pifont}
\usepackage{color}
\usepackage{listings}
\usepackage{pdfpages}
\usepackage{array}
\usepackage{algorithm}
\usepackage{algorithmic}
\usepackage{float}
\usepackage{caption}
\usepackage{lastpage}
\usepackage{afterpage}
\usepackage{lipsum}
\usepackage{adjustbox}
\usepackage{cite}
\usepackage{enumitem}
\usepackage[cmex10]{amsmath}
\usepackage[utf8]{inputenc}
\usepackage{tabularx}
\usepackage{lscape}
\usepackage{epstopdf}
\captionsetup{belowskip=12pt,aboveskip=8pt}
\newcommand{\tab}{\hspace*{2em}}
\graphicspath{{Figures/}}
\DeclareGraphicsExtensions{.pdf,.png,.jpg}
\usepackage[utf8]{inputenc}
\DeclareUnicodeCharacter{00A0}{~}

% End of Latex Packages

% End of personal stuff
%
% Personal Information 
% ----------------------------
%
% Please check this part and fill in information about your thesis
%
% Name and Surname
\author{ERENCAN DUYMAZ}
% Thesis Title English and Turkish
\title{Investigation of the Limits of Inertial Support Implementation in Variable Speed Wind Turbines and the Effects of Synthetic Inertia in Power System Frequency Stability}
\turkishtitle{Değişken Hızlı Rüzgar Türbinlerinde Atalet Desteğinin Sınırlarının İncelenmesi ve Yapay Atalet Desteğinin Güç Sistemleri Frekans Stabilitesine Etkileri}

\date{December 2018}
 
% prof : Prof. Dr.
% assocprof : Assoc. Prof. Dr.
% assistprof : Assist. Prof. Dr.
% dr : Dr.
%
% Director of Institute
\director[prof]{Gülbin Dural Ünver}
% Head of Department
\headofdept[prof]{Bölüm Başkanı}
%
% Supervisor : English and Turkish
\supervisor[Ass.Prof.Dr.]{Ozan KEYSAN}
% \turkishsupervisor{  } %if you will hard-code the academic title
%
% Affiliation of Supervisor in English and possibly in Turkish
\departmentofsupervisor{Department of Electrical and Electronics Engineering, METU}
%
% Committee Members
% In general members are sorted according to their academic titles
%
% Proffesors (1)
% Associate Professors (2)
% Assistant Professors (3)
% Other (4)
% 
% IMPORTANT:  All affiliatons should fit in a single line
% If affiliation line is broken into two lines you should shorten the affiliation by using 
% abbrevations or any other means
%
% First committee member should be the chair of examining committee
% Typically the chair is one of the highest ranked committee members
% Ask your supervisor if you are not sure
\committeememberi[prof]{Jüri}
\affiliationi{JüriBölüm, METU}
% Second committee member is always your supervisor
\committeememberii[prof]{Jüri}
\affiliationii{JüriBölüm, METU}
% If you are an M.Sc. student and your Co-Supervisor is in your 
% examination committee, then third committee member is always your co-supervisor
%
% IMPORTANT: If you are Ph.D. student your co-supervisor can not be in your 
% examination committee.
\committeememberiii[prof]{Jüri}
\affiliationiii{JüriBölüm, METU}
% Fourth committee member
\committeememberiv[assocprof]{Jüri}
\affiliationiv{JüriBölüm, METU}
% Fifth committee member
\committeememberv[assistprof]{Jüri}
\affiliationv{JüriBölüm, Ankara University}
%
% Keywords : English & Turkish, Comma seperated
\keywords{Power System Frequency Stability, Inertial Support, Synthetic Inertia, Virtual Inertia, Renewable Energy}
\anahtarklm{Güç Sistemleri Frekans Stabilitesi, Atalet Desteği, Yapay Atalet, Sanal Atalet, Yenilebilir Enerji}
%
% Abstract in English
%
\abstract{The share of renewable energy systems in the installed capacity increases dramatically. Research shows that increase in the renewable energy percentage brings power system stability problems due to the decrease in the grid aggregated inertia. To prevent the decrease in the grid inertia, renewable energy systems should possess the inertial support capability that is the inherited behaviour of the conventional synchronous generators. Especially, variable speed wind turbines are able to provide additional amount of active power in the frequency disturbances by extracting the kinetic energy stored in the turbine and generator inertia. In this study, the inertial support implementation is studied for variable speed wind turbines with a full-scale power electronics. To increase the active power as desired, Machine Side Converter is modified with an additional control loop. In the first part of the thesis, active power of the wind turbine is increased to the limits and the maximum achievable active power is found out to be restricted by the wind speed. It is also found that the wind turbine can increase its output power by 40\% of rated power in the low and medium wind speeds. Moreover, even though the high speed scenarios gives limited increased power, it does not require any speed recovery state. The probability of different wind speeds and the inertial supports are found according to the wind speed measurement taken from field. In the second part of the thesis, the synthetic inertia implementation is presented by the provision of inertial support proportional to rate of change of frequency. The effect of the implementation in the P.M.Anderson test case is observed for different inertia constants. It is discovered that the effect of renewable penetration in the frequency stability is negligible when the synchronous generators are kept in the operation. Nonetheless, frequency stability in the test system gets more vulnerable with renewable energy penetration when the conventional generators are decommissioned by the economical concerns. In this case, the synthetic inertia implementation with different inertia constants possess the ability to lower RoCoF following a frequency disturbance. }
%
% Turkish Abstract
%
\oz{Kurulu güçteki yenilenebilir enerji oranı önemli ölçüde artmaktadır. Araştırmalara göre yenilenebilir enerji oranındaki artış şebekenin toplu ataletindeki düşüşten kaynaklanan güç sistemleri stabilite sorunlarını da beraberinde getirmektedir. Şebeke ataletindeki bu düşüşü engellemek için, yenilenebilir enerji sistemleri konvansiyonel senkron makinelerin doğuştan sahip olduğu atalet desteği yeteneğine sahip olmalıdır. Özellikle, değişken hızlı rüzgar türbinleri frekans bozunum anlarında, türbin ve generatör ataletinde depolanmış hareket enerjisini soğurarak şebekeye fazla güç basabilir. Bu çalışmada, tam ölçek güç elektronikli değişken hızlı rüzgar türbinleri için atalet desteği uygulanmıştır. Aktif gücü istenildiği gibi arttırabilmek için Makine Tarafı Kontrolcüsüne kontrol döngüsü eklenmiştir. Bu tez çalışmasının ilk kısmında, rüzgar türbinin aktif gücü sınır noktalarına kadar arttırılmıştır ve ulaşılabilir maksimum gücün rüzgar hızıyla kısıtlandığı gözlenmiştir. Ayrıca, düşük ve orta rüzgar hızlarında rüzgar türbininin aktif gücünü nominal gücünün \%40'ı oranda arttırabildiği gözlenen bulgular arasındadır. Bunun yanında, yüksek rüzgar senaryolarında sınırlı güç artışı gözlense de, generatör hızının toparlanma evresine gerek duymadığı gözlenmiştir. Sahadan alınan rüzgar hızı verilerine göre değişik rüzgar hızı ve atalet desteklerinin olma olasılıkları da hesaplanmıştır. Tezin ikinci kısmında, frekansın değişim hızıyla ortantılı atalet desteği uygulayarak yapay atalet desteği uygulanmıştırç Bu uygulamanın etkileri, değişik atalet sabitleri ile P.M. Abderson test düzeneğinde gözlenmiştir. Buna göre, konvansiyonel senkron makineler operasyonda olduğu sürece, yenilenebilir enerji penetrasyonun frekans stabilitesine olan etkisinin ihmal edilebilir düzeyde olduğu görülmüştür.Ancak, generatörlerin ekonomik kaygılarla operasyondan alındığı durumlarda, yenilenebilir enerji penetrasyonun frekans stabilitesini zayıflattığı gözlenmiştir. Bu durumlar, değişik atalet sabitleriyle uygulanabilen yapay ataletin frekans bozunumlarında sistemin frekans değişim hızını azalttığı sonucuna varılmıştır.} 
%
% Dedication 
\dedication{\textit{To my precious mom...}}
%
%
% Acknowledgements   
\acknowledgments{Teşekkür edilecekler}
%
% End of Personal and Introductory Information
%%%%%%%%%%%%%%%%%%%%%%%%%%%%%%%%%5
\begin{document}
% Preliminaries
\begin{preliminaries}
% If you are willing to use any custom stuff before Chapters, put it here
% Such as List of Abbreviations
% Check the abbreviations.tex for a template list of abbreviations

\begin{theglossary}{LONGESTABBRV}
\item[AGC] Automatic Generation Control
\item[DFIG] Doubly Fed Induction Generator
\item[PMSG] Permanent Magnet Synchronous Generator
\item[MSC] Machine Side Converter or Controller
\item[GSC] Grid Side Converter or Controller
\item[LSC] Line Side Converter or Controller
\item[AC] Alternating Current
\item[DC] Direct Current
\item[MOSFET] Metal Oxide Semiconductor Field Effect Transistor
\item[IGBT] Insulated Gate Bipolar Transistor
\item[PI] Proportional-integral
\item[LVRT] Low Voltage Ride-Through

\end{theglossary}


\begin{theglossary}{}
	
	\begin{itemize}[leftmargin=4.5em,align=parleft,labelsep=1cm]
		
		\item[$C_{p}$] 			Power Coefficient
		\item[$f_{grid}$] 		Grid Frequency
		\item[$H$] 				Inertia Constant
		\item[$J_{tur}$] 		Turbine Inertia
		\item[$\lambda$] 		Pitch or Blade Angle
		\item[$P_{e}$] 			Electromechanical Output Power
		\item[$P_{m}$] 			Input Mechanical Power
		\item[$P_{tur}$] 		Turbine Active Power
		\item[$P_{gen}$] 		Generator Active Power
		\item[$p_{f}$] 			Number of Pole
		\item[$S_{base}$] 		Base Apparent Power
		\item[$\omega_{m}$] 	Generator Speed
		\item[$\omega_{max}$] 	Maximum Generator Speed
		\item[$T_{Plim}$]		Torque Limited by Active Power of Wind Turbine
		\item[$T_{Slim}$] 		Torque Limited by Apparent Power of Wind Turbine
		
		
	\end{itemize}
	
\end{theglossary}
% End of Preliminaries
\end{preliminaries}
%   
% Latex content Goes Here 
% 
%

\setlength{\parindent}{0em}
\setlength{\parskip}{10pt}


\chapter{INTRODUCTION}
\label{chp:1}
\section{Global Renewable Energy Status}
The share of the renewable energy systems has been reached significant levels. At the end of 2017, the renewable power capacity has reached 2179 GW throughout the world including hydro power plants \cite{InternationalRenewableEnergyAgencyIRENA2018}. Fig. \ref{renewablecap} shows the installed renewable energy capacity for leading countries at the end of 2016 and 2017. China, USA, Brazil and Germany constitutes almost half of the world total capacity. China has the biggest installed renewable capacity so far and increased its capacity by 73 GW in 2017.\par
\begin{figure}[h]
	\centering
	\includegraphics[scale=0.55]{renewablecapacities.pdf}
	\caption{Installed Renewable Energy Capacity of Leading Countries \cite{InternationalRenewableEnergyAgencyIRENA2018},\cite{InternationalRenewableEnergyAgency2017}}
	\label{renewablecap}
\end{figure}
EU countries promotes the renewable energy systems from the very beginning. In 2008, 20 20 by 2020-Europe's Climate Change Opportunity report has been released by EU Commission and two key targets are set for 2020 \cite{EuropeanCommission2008}: 
\begin{itemize}  
	\item At least 20 \% reduction in greenhouse gases (GHG) by 2020
	\item Achieving 20\% renewable energy share in energy consumption of EU by 2020
\end{itemize}
In order to accomplish this target, the Renewable Energy Directive is published in 23 April 2009. This directive has set national binding targets for EU countries in order to accomplish the 20\% renewable energy target for EU and 10 \% target for the renewable energy usage in the transport. \cite{EuropeanParliament2009} As a result, each EU country has been determined their national action plans. In order to achieve the 20 \% target, each member state determine their own targets ranging from 10\% in Malta to 49\% in Sweeden. According to the latest release by Eurostat, renewable share of the EU in energy consumption has reached 17 \% in 2016 \cite{States2016}. Moreover, eleven of EU member states has already achieved their 2020 targets.\par
\begin{figure}[h!]
	\centering
	\includegraphics[scale=0.55]{windcapacity.pdf}
	\caption{Wind Power Capacity of Leading Countries in 2016 \cite{InternationalRenewableEnergyAgencyIRENA2018},\cite{InternationalRenewableEnergyAgency2017}}
	\label{windcap}
\end{figure}
Wind power has the highest share among the renewable energy sources in the installed renewable energy capacity except for hydro power. The wind power capacity at the end of 2017 has reached 514 GW worldwide\cite{InternationalRenewableEnergyAgencyIRENA2018}. The wind power capacity of the leading countries is shown in the Fig. \ref{windcap}. As in the case of total installed renewable energy capacity, China and USA have also the highest installed capacities in the wind power capacity. \par
\section{Global Renewable Energy Future}
The share of renewable energy is increasing continuously. Today, the discussion is about whether 100\% renewable energy is possible in the upcoming future. In \cite{REN212017d}, grid integration issues of wind and solar and the lack of sufficient storage technologies are considered as the main barrier for this target meanwhile the major problem seems as the existing energy industry. Nonetheless, a significant renewable share is expected even though the 100\% is reality or not.\par
The renewable energy reports estimate the share of renewable energy in the total energy consumption for 2030 and 2050. Figure \ref{EU2030} shows the EU renewable energy share for 2030. Moreover, the report published by IRENA (International Renewable Energy Agency) estimates the share of renewable energy in EU as 24\% by 2030 which is below proposed target of 27\%\cite{IRENA2014}.\par
\begin{figure}[h!]
	\centering
	\includegraphics[scale=0.35]{EU2030.png}
	\caption{Renewable energy share in total energy consumption by EU for 2015, 2020 targets and 2030 potential according to REmap \cite{EuropeanCommission2018}}
	\label{EU2030}
\end{figure}
\begin{figure}[h!]
	\centering
	\includegraphics[scale=0.3]{2030map2.png}
	\caption{Renewable energy shares for 2010, 2030 Reference Case and 2030REmap \cite{IRENA2014}}
	\label{2030map}
\end{figure}
Renewable shares of REmap countries in 2010, 2030 reference case and 2030REmap and the world average are also shown in Fig.\ref{2030map}. 
\section{Renewable Energy Problems}
It is an undeniable fact that renewable energy systems are advantageous in terms of global warming and carbon dioxide emission. Nonetheless, they also have disadvantages to the system operators due to intermittent energy generation profile. First of all, the term intermittent in the literature is related to the variable and uncontrollable nature of the renewable sources \cite{KlingeJacobsen2010}. Since the source of the RES is variable, it is not possible to adjust its output according to the demand. Therefore, the thermal plants have to be in the operation when high wind speeds and solar radiation exist. Moreover, the system requires additional start-ups and rise from partly loaded plants in order to balance the energy in the system because of the uncertainty of RES. These all create additional costs caused by high share of RES in the system \cite{Zipf2013}. Besides, power grid will face with transmission system issues as overloaded transmission lines, changes on the protection and control in the distribution system, greater level of power-factor control and low voltage ride-through (LVRT) requirements when the RES share is increased in the grid \cite{Ipakchi2009}.\par
Another challenge of increasing RES is the problem of power system frequency stability. Since the frequency of the power system depends on the balance between generation and consumption, grid operators are responsible for adjusting the generation in order to maintain a constant frequency. However, the renewable energy generation is strictly dependent on the renewable source i.e. solar radiation or wind speed. Therefore, renewable systems makes the system operation harder due to their intermittent and uncertain power generation profiles. Moreover, as the renewable systems with power electronics interface increase in the electricity grid, the grid equivalent inertia decreases. In \cite{Gautam2011}, the reduced grid inertia due to the high DFIG wind turbine penetration is emphasized. Moreover, the results of the reduced grid inertia following a disturbance is listed as: 
\begin{itemize}
	\item increased effective aggregated angular acceleration of synchronous machines which require high restoring forces
	\item high rate of change of frequency and hence, decreased frequency nadir
\end{itemize}
It should be noted that this problem is not specific to DFIG wind turbines but renewable energy systems which are connected to grid with power electronics. Conventional synchronous generators rotate at synchronous speed which is proportional to the grid frequency. If the grid frequency decreases, then the synchronous speed also decreases. In this case, the generator active power is increased inherently due to kinetic energy extraction from the generator inertia. The increase in active power provides action time for primary controllers and crucial for frequency stability. \par
\begin{figure}[h!]
	\centering
	\includegraphics[scale=0.75]{windgentop.png}
	\caption{Wind Turbine Generator Configurations \cite{Muljadi2012}}
	\label{windtop}
\end{figure}
Different turbine topologies gives different reactions to the frequency disturbances. Wind turbine generator topologies are shown in Fig. \ref{windtop}. Type-1 turbines are connected to grd with a asynchronous generators. The wind turbine generates active power as turbine rotates faster than synchronous speed. Therefore, the generator operates at the linear part of the torque-slip curve. Hence, the change in the grid frequency causes smaller decrease in the turbine speed. Type-2 is very similar to Type-1 except for the variable resistor which can shift the torque speed curve slightly. Hence, the frequency deviations affects the active power output of Type-1 and Type-2 \cite{Muljadi2012}. Type-3 wind turbines include DFIG whose stator is directly connected to grid meanwhile the rotor is connected to grid with a power electronics. Even though the stator is directly coupled to stator, the power electronics enable wind turbine to operate in a range of speeds. Therefore, the rotor frequency is also decoupled from grid. Type-4 wind turbines are connected to grid with back-to-back converters. This is why Type-3 and 4 wind turbines is not affected from the grid frequency deviations. Therefore, these systems have no contribution to the grid inertia whether the system includes inertia or not. Hence, the aggregated grid inertia is reduced with the penetration of wind turbines with power electronics. The comparison for different type of generators is made in \cite{VanDeVyver2016} and listed in Table \ref{generatorcomparison}.\par  
\begin{table}[h!]
	\centering
	\begin{tabular}{lc}
				\hline
		\multicolumn{1}{c}{\textbf{Type of the generator}}                                                                            & \textbf{Inertial Response Behaviour} \\ \hline
		Conventional Synchronous Generator                                                                                            & ++                                   \\
		Fixed Speed Induction Generator (FSIG)                                                                                        & +                                    \\
		Doubly Fed Induction Generator (DFIG)                                                                                         & -                                    \\
		\begin{tabular}[c]{@{}l@{}}Variable Speed Wind Turbine Generator\\ (Connected with Full Scale Power Electronics)\end{tabular} & None                                 \\ \hline
	\end{tabular}
	\caption{Comparison of Different Type of Generators for Inertial Response Behaviour}
	\label{generatorcomparison}
\end{table}
Another reason for the decrease in the grid inertia is the de-commitment or dispatch of the conventional sources due to economic concerns. Since the renewable energy has the lowest cost for energy production, it preferred instead of conventional generators. As a result, conventional generators are dispatched to a lower generation profile or taken-off from operation.\par
Note that grid inertia is directly related to the amount of load in the system in addition to the share of RES. Therefore, the amount of online generator flactuates within time. Hence, the scenario in which the system has low demand and also high renewable generation is the most critical one since the lowest grid inertia will be faced in the network.
\section{Literature Review}
Studies regarding inertial support date back to early 2000s. In the study \cite{Lalor2004}, the effect of the increasing wind energy penetration has been investigated. The study concludes that increasing share of wind energy increases the primary reserve requirement for the successful grid operation. The increased frequency deviations, especially in light load conditions (high wind generation with low consumption scenario) can be mitigated in the system as long as the wind generation provides inertia support. Study in \cite{Ekanayake2003} states that DFIG wind turbines are de-coupled from power system resulting in no contribution to system inertia. A supplementary loop is proposed for reinstating the machine inertia. Moreover, in \cite{Ekanayake2004}, performance of the  supplementary control loop is evaluated with the comparison of the inertial support of a fixed-speed wind turbine. The proposed control loop has been validated in \cite{Morren2006} and compared with the droop control in \cite{Morren2006a}. \par 
It is an undeniable fact that renewable energy systems are the most economical way of producing electrical energy due to absence of any fuel cost. Therefore, they are to be operated in their rated power. However, they have to curtail their power in order to leave a margin for droop control. Droop control by wind energy is also studied in the literature. In \cite{Muljadi2012}, the inertial support of different type of wind turbines is compared. It is concluded that the Type-4 wind turbines are able to perform better performance for inertial support due to the power electronics interface. Moreover, combination of inertial support and droop control produces better results in these wind turbines.\par
Fast inertial response is studied in the literature as Torque-Limit based inertial support or Stepwise Inertial Control in the studies \cite{Wang2016b}, \cite{Wang2016}. Nonetheless, the support is achieved by the operation in the limit torque independent from the size of the disturbance and the support is ended at the pre-defined generator speed. Still, limits of the support for varying wind speed is not studied as well as the restoration of the generator speed is not discussed. \par
The concept of the synthetic inertia has been widely studied in the literature since it is a method for renewable energy systems to emulate synchronous generators. In \cite{Zhu2013a}, the method is implemented a VSC-HVDC transmission systems in order to improve frequency stability of a weak grid. Study \cite{Hernandez2017} focuses on the implementation in the PV systems in a coordination with energy storage systems. In the studies \cite{VanDeVyver2016}, \cite{Conroy2008}, the method is implemented on a variable speed wind turbines. However, the capability of the wind turbines are not studied in the literature. 
Practical limits of the inertial support has been studied in \cite{Gonzalez-Longatt2016} by varying the inertia constant to be emulated. However, the practical limits in terms of maximum achievable power and turbine internal parameters are not focused in the study. Moreover, the studies does not compare the two main method of the inertial support namely, fast frequency support and frequency based method. Finally, the studies does not focus on the wind speed for the inertial support capacity. 

\section{Thesis Motivation}
The frequency of the electric grid depends on the balance between generation and consumption. Grid operators are responsible for maintaining this balance so that frequency of the grid is maintained between allowed dead-band. In order to achieve this purpose, power generation is adjusted according to the consumption value. However, the balance between supply and demand might be disturbed with unintentional generator trip or instant load connections. Grid frequency decreases such instants until the generation is increased to arrest the frequency. Inertia of the electric grid provides additional power from the stored kinetic energy and avoid the system frequency from decreasing down very fast. That is called as inertial support and it is very important for power system frequency stability.\par
Although renewable energy systems are beneficial for environmental concerns and lower energy cost, higher renewable penetration also brings operational challenges for system operators. One of the most important problem that comes with renewable energy is the power system frequency stability. With the high renewable penetration, grid aggravated inertia decreases. As a result, grid frequency deviates steeper for disturbances. To avoid steeper frequency declines in the grid, all generation technologies should provide inertial support for the frequency disturbances.\par
Wind energy systems, especially variable speed wind turbines with full scale power electronics are the most promising renewable energy systems that can contribute to grid frequency stability thanks to their high inertia in their blades and generator and also their back-to-back converters that give ability to control its active power. Therefore, wind energy conversion systems are required to participate in ancillary services for frequency stability in order to reach a stable power system network in the upcoming future. \par
In this study, the limits of the inertial support is investigated in the different wind speeds. By considering the wind speed status, the potential of the wind turbine is revealed for the frequency stability analysis. Moreover, fast frequency response and synthetic inertia support is both implemented on a variable wind speed turbine to investigate their contribution to electricity grid in the frequency disturbances. 

\section{Thesis Outline}
This thesis study focuses on the wind turbine inertial support limits and its effects in power system stability. The thesis starts with a brief summary of the renewable energy status in Chapter \ref{chp:1}. By reviewing the share of the renewable energy systems and the targets for upcoming future highlight the importance of the frequency stability studies.\par
In the Chapter \ref{chp:2}, the frequency concept in power systems is extensively described. Since renewable energy systems are replaced or preferred over the conventional power generation units, the electricity grid is facing with frequency stability issues due to the absence of inertia-less units. Therefore, the behaviour of old-fashion power plants are described under frequency disturbances. Moreover, the frequency regulating mechanisms are presented. Finally, the energy markets are also explained in order to emphasize the role of renewable energy systems.\par
Chapter \ref{chp:3} presents the modelling of wind turbine used in this study. Since the existing variable speed wind turbines require modification in order to integrate to electricity grid, detailed modelling of these wind turbines is presented. By utilizing synthetic inertia method, a relation between grid frequency and the active power output is constructed.\par
The limits of the active power increase is investigated in Chapter \ref{chp:4}. The ability of increasing its active power output is already presented in the literature. However, the limits of support power is studied for different wind speed scenarios. Maximum achievable active power are studied for varying wind speeds. The real wind speed measurements from site are utilized to find to probability of support power for different wind speeds is calculated. Wind turbine inertial support limits and turbine internal parameters are observed for the non-dynamic frequency response.\par
Synthetic inertia implementation within a wind farm is studied in the Chapter \ref{chp:5}. The effect of synthetic inertia is observed in a dynamic test case with renewable penetration.Test case is modified with different combinations in which the system is penetrated with wind farm with/without generator decommission are studied in this chapter. Frequency response of the test system is tested for different grid configurations as well as different emulated inertia constants.\par
Chapter \ref{chp:6} presents a basic conclusions for the inertial support whose implementation is either frequency dependent or not. Moreover, an economical analysis from an energy provider perspective is given. Two different payment methods are constructed and compared to estimate which economical motive that can persuade the energy provider for participating grid supporting methods with renewable energy systems.  



















% CHAPTER 1
\chapter{POWER SYSTEM FREQUENCY STABILITY}
\label{chp:2}
The increasing renewable energy penetration deteriorates the power system frequency stability. One of the most severe effect of renewable energy penetration is the reduction in grid inertia. Grid aggregated inertia is important for the frequency stability. As the frequency falls, a part of the kinetic energy is extracted from the grid inertia and released inherently from synchronous generators to grid. Therefore, as the grid aggregated inertia decreases, the control of the frequency becomes difficult resulting in quick changes in the grid frequency.\par
In this chapter, the internal dynamics of the power system will be presented in terms of frequency stability. The frequency regulation mechanisms in the grid are also given. Furthermore, the energy market is briefly explained in order to understand the role of renewable energy providers in the market.
\section{Synchronous Generator and Synchronous Speed}
Synchronous machines produce torque only in synchronous speed. If a transient over-speeding occurs in the rotor, the torque cannot be produced that results in a deceleration. This makes the rotor angular velocity to strictly coupled to the electrical frequency  of the stator rotating MMF. Moreover, these machines are equipped with damper windings which are basically induction machine windings. If the frequency of  grid changes, damper windings create a torque which creates a force to synchronize the speed to the grid frequency. \par
\begin{equation}
n_{s}=\frac{120f}{p}
\label{syncspeed}
\end{equation}
Relation between grid frequency and the synchronous speed is given in Eq. (\ref{syncspeed}) in terms of rpm where $n_{s}$ is the synchronous speed in rpm, $f$ is the grid frequency in Hz, $p$ is the number of poles of the generator \cite{Kundur}.

\section{Swing Equation}
\label{swing}
Rotational speed of synchronous machines changes according to the net torque acting on the rotor. Therefore, the speed is maintained constant unless there is a difference between mechanical and electromechanical torque. The equation of motion is given in Eq. (\ref{eqmotion}) where $J$ is aggravated moment of inertia of the generator and the turbine in $kgm^{2}$, $\omega_{m}$ is the rotor angular velocity in $rad/s$, $T_{m}$ and $T_{e}$ are mechanical and electromechanical torques in $Nm$. $T_{a}$ is the accelerating torque in $Nm$ which determines the acceleration or deceleration in the rotor according to its sign.
\begin{equation}
J\frac{d\omega_{m}}{dt}=T_{m}-T_{e}=T_{a}
\label{eqmotion}
\end{equation}
In power system network, the power ratings of the generators in operation and corresponding moment of inertia values vary. Inertia constant is defined as the ratio of kinetic energy stored in the inertia to the power rating of the generator as in Eq. (\ref{inertiaconstant}) where $\omega_{0m}$ denotes the rated angular velocity of generator in $rad/s$ and $S_{base}$ is the rated apparent power in VA. $H$ indicates the time duration in which generator produces its rated apparent power by only using its kinetic energy in the inertia. Thus, $H$ is a better indication of factor for power system frequency stability analysis compared to $J$. Hence, it is more convenient to use inertia constant, $H$ which varies between 2 and 9 seconds \cite{Kundur}.
\begin{equation}
H=\frac{{\frac{1}{2}}J\omega_{0m}^{2}}{S_{base}}
\label{inertiaconstant}
\end{equation}
Substituting Eq. (\ref{inertiaconstant}) into Eq. (\ref{eqmotion}) and replacing units to per-unit quantities yield the swing equation given in Eq. (\ref{eqmotion5}) where $\overline{P_{m}}$ is the input mechanical power in pu and $\overline{P_{e}}$ is the electromechanical output power in pu. It defines the inherent behaviour of a synchronous generator against the frequency deviations in the grid. When the grid frequency falls, a subsequent decrease in the rotor speed is observed. According to Eq. (\ref{eqmotion5}), a negative term is found in the left-hand side. This means that rotor electromechanical output power, $\overline{P_{e}}$ is increased inherently. It should be noted that the additional energy is not taken from the input mechanical power but it is extracted from the kinetic energy. Moreover, the decreased energy is injected to grid whenever the frequency increases due to the acceleration in the rotor.
%\begin{equation}
%J=\frac{2H}{\omega_{0m}^{2}}{S_{base}}
%\label{inertiaconstant2}
%\end{equation}
%\begin{equation}
%\frac{2H}{\omega_{0m}^{2}}{S_{base}}\frac{d\omega_{m}}{dt}=T_{m}-T_{e}
%\label{eqmotion2}
%\end{equation}
%\begin{equation}
%\frac{2H}{\omega_{0m}^{2}}{S_{base}\omega_{m}}\frac{d\omega_{m}}{dt}=P_{m}-P_{e}
%\label{eqmotion3}
%\end{equation}
%\begin{equation}
%2H\frac{\omega_{m}}{\omega_{0m}}\frac{d(\omega_{m}/\omega_{0m})}{dt}=\frac{P_{m}-P_{e}}{S_{base}}
%\label{eqmotion4}
%\end{equation}
\begin{equation}
2H\overline{\omega_{m}}\frac{d\overline{\omega_{m}}}{dt}=\overline{P_{m}}-\overline{P_{e}}
\label{eqmotion5}
\end{equation}
\section{Frequency in Power Systems}
The frequency in a power system is related to the speed of the synchronous generators and changes according to the swing equation. The frequency of the each generator is not the same in the network since each generator does not have the same speed. Nonetheless, the fluctuations in the generator speeds are called rotor swings and can be negligible in the steady state. Hence, the network can be assumed as a single generating unit by neglecting small differences between the generator speeds. The swing equation basically investigates the relation between mechanical and electromechanical powers and the rate of change of angular speed of a generator. However, it is also applicable to grid in order to estimate the grid frequency.\par
\begin{equation}
\label{systemswing}
2H_{sys}\overline{f}_{sys}\frac{d\overline{f}_{sys}}{dt}=\overline{P}_{tm}-\overline{P}_{te}
\end{equation}
If the generators of the grid are considered as a single generator, the inertia of the equivalent generator is aggregated from each generator in the network. In this case, average frequency in the network can be found as in Eq. (\ref{systemswing}) where $P_{tm}$ is the aggravated mechanical input power of the generators meanwhile $P_{te}$ is the aggravated electromechanical output power. In other words, the system frequency depends on the balance between generation and consumption. It should be noted that generation means the input mechanical power of the generators meanwhile the demand is absorbed from the electromechanical output power of the generators. Hence, the difference between these causes either acceleration or deceleration in the grid frequency. \par
\begin{figure}[h!]
	\centering
	\includegraphics[width=.9\linewidth]{frequencypool.pdf}
	\caption{Frequency Behaviour in Electric Grid with the Water Level in a Container Analogy \cite{Eto2010}}
	\label{frequencyingrid}
\end{figure}
The behaviour of the frequency in electric grid is depicted in Fig. \ref{frequencyingrid}. As it can be seen from the water level in a container analogy, the frequency of the system is dependent on the in-flow and the out-flow. Therefore, in the electricity grid, frequency increases as the aggregated input power is higher than the aggregated output power. Note that, the direction of the frequency is dictated by this balance. Having a constant 49.8Hz frequency does not mean that consumption is higher than generation. If the frequency is constant, then the input mechanical power is equal to output power.\par
The variation of the grid frequency is depicted for a typical day in Fig \ref{06decfreq}. The frequency deviates continuously during the day. It should be noted that there exist hourly peaks in the frequency. The peaks occur due to the change in the hourly generation shift according to the unit commitment. Since the generation level and generating units are changed for the next hour, the frequency deviates hourly in the grid.
\begin{figure}[h!]
	\centering
	\includegraphics[width=.9\linewidth]{06decfreq.pdf}
	\caption{Variation of the Frequency in a Typical Day (06 Dec 2018) \cite{teiasfreq}}
	\label{06decfreq}
\end{figure}
\section{Frequency Regulating Mechanisms}
Having a constant frequency is one of the most important responsibilities of a system operator. In order to have a constant frequency, supply is being adjusted according to the demand continuously. Consequently, the system frequency varies between a band-gap. The variation in the grid frequency depends on the disturbances which are generally a sudden generation outage or instant load connection. The size of the disturbance determines the severity of the frequency change and there are three main mechanisms to arrest the frequency changes in the system. \par
The frequency control services in England and Wales are depicted in Fig. \ref{freqcontrol} for a frequency disturbance. The frequency is maintained between 49.8Hz and 50.2Hz for continuous service. A frequency disturbance event causes frequency to decline. However, the decline in the frequency is arrested with the help of the inertial support of the conventional generators and the primary frequency controllers. The main responsibility of the primary frequency control is arresting the frequency decline. Restoration of grid frequency to nominal is the responsibility of the secondary and tertiary frequency controls.
\begin{figure}[h!]
	\centering
	\includegraphics[scale=0.35]{freqq.png}
	\caption{Grid Frequency Control in England and Wales \cite{Ekanayake2008}, \cite{Erinmez1999}}
	\label{freqcontrol}
\end{figure}
\subsection{Primary Frequency Control}
Following a generator outage or a sudden load connection event, frequency starts decreasing. The rate of change of the frequency is dependent on the severity of the event by means of power unbalance and the available inertia of the power system. Such frequency disturbance requires increased input power. However, the increase in the input mechanical power should be activated very fast and should be automated. This responsibility is assigned to generating units with primary frequency control. The active power generation of these units is increased or decreased by the governor depending on the network frequency direction. Notice that each generator in the power system does not necessarily perform primary frequency control function. In this case, their active power generation is independent from the network frequency. The contribution to primary frequency control is a responsibility but also a way to sell higher energy to grid operator.\par
The primary frequency control is automated with droop control defined in the generator speed governors. According to droop control, the generator power should be increased according to the frequency deviation from nominal. Therefore, the generating unit does not utilize its whole capacity but rather keeps a capacity which is called as spinning reserve. According to the droop curve, the generation unit should increase its output power no longer than 15 seconds and keep its operation up to 30 minutes\cite{Machowski2011}.
\subsection{Secondary Frequency Control}
The frequency is recovered back to nominal value with the Secondary Frequency Control action. This controller might be a single or multiple centers that monitor the frequency and adjust the generation accordingly. They are also called as Automatic Generation Control (AGC) systems and their action takes a few minutes. AGC monitors the frequency deviation from the nominal and takes action to recover frequency back to nominal. With the secondary frequency control action, primary controllers decrease their production back to their pre-disturbance value.
\subsection{Tertiary Frequency Control}
The final frequency control mechanism is the Tertiary Frequency Control. If the frequency is not recovered back to nominal value with the secondary controllers, tertiary frequency controllers manually activate the load shedding which is an undesired situation by the network operator. However, it is an emergency case which might result in black-out and requires immediate action.
\section{Energy Market}
Since the energy is generated and distributed by private energy companies, a system operator should be responsible for maintaining a balance in the power network. The frequency is kept inside the operational band in the electricity network by balancing the supply and the demand by intersecting the supply and demand curves inside the different time intervals. In this way, the balance is ensured in the market by day ahead, intra-day and balancing markets. 
\subsection{Day Ahead Market}
The load power in a network has a distinct characteristic depending on the day of the week or the hour of the day. By foreseeing the next day demand power variation, the electricity market collects the bids from the energy suppliers and consumers. According to submitted bids, the next day generation price is determined by intersecting the supply and demand price curves. The price of the energy is called Market Clearing Price (MCP). These bids are submitted for the next day and the prices are determined before the corresponding day.
\subsection{Intra-Day Market}
Even though the estimations for the upcoming day load power has significant accuracy with the advanced estimation methods, networks are subjected to unexpected problems such as generator trips, line outages. Therefore, intra-day market contributes the balance of the market between the day ahead market and balancing market. Moreover, it gives the participants almost real-time trading opportunity meanwhile it increases the sustainability of the market. After day ahead market has closed for the corresponding day, the bids are submitted to system. In other words, MCP is already determined for the corresponding day meanwhile the rest of the day prices are not set. The average of the bids submitted to market is called as Weighted Average Price (WAP).
\subsection{Balancing Market}
Primary and secondary control reserves are maintained in the system in order to improve the balance for the instant deviations in the frequency. The frequency is first arrested by the primary controllers and it is restored by the secondary controllers. The generation units that participate primary and secondary control promise a defined generation capacity to these actions. Balancing market is much more different than day ahead and intra-day market since its main goal is the network security rather than electricity trading. The price of the energy in this market called as System Marginal Price (SMP).\par
The price of the energy changes according to the market type. In Fig. \ref{markets}, three market prices such as Market Clearing Price (MCP), Weighted Average Price (WAP) and System Marginal Price (SMP) are shown for a typical day. MCP and WAP are close to each other since they are scheduled much before operation time. Nevertheless, SMP fluctuates more since it is the resultant price of the balancing market. Notice that balancing market price might be higher than day ahead and intra-day prices. However, balancing market is not appropriate platform for trade purposes due to the fact that main object in this market is the successful operation of the system.
\begin{figure}[h!]
	\centering
	\includegraphics[scale=0.7]{marketprices.pdf}
	\caption{Energy Prices on 06 Dec. 2018 \cite{TEIAS2019}}
	\label{markets}
\end{figure}
\subsection{Feed-In Tariff}
\label{section-price}
Significant amount of energy produced inside the Turkish electricity network is based on exported sources such as coal and gas. As a result of this, the energy sector is highly dependent on foreign countries. In order to decrease the dependency on the external sources, the renewable energy sources are supported by government in Turkey. The energy generated by renewable energy systems are sold with the feed-in tariff (FIT) within the pre-determined time interval according to the purchase agreement. This decreases the return of investment due to the fact that all produced energy will be sold during this period with a remarkable price. The feed-in tariff for different renewable energy systems is listed in Table \ref{price}.\par
\begin{table}[h]
	\centering
	\begin{tabular}{cc}
		\hline
		\textbf{Renewable Energy System} & \textbf{\begin{tabular}[c]{@{}c@{}}Feed-In Tariff\\ (cent/kWh)\end{tabular}} \\ \hline
		Hydro                            & 7.3                                                                          \\
		Wind                             & 7.3                                                                          \\
		Geothermal                       & 10.5                                                                         \\
		Biomass                          & 13.3                                                                         \\
		Solar                            & 13.3                                                                         \\ \hline
	\end{tabular}
\caption[Feed-In Tariff for Renewable Energy Systems in Turkey]{Feed-In Tariff for Renewable Energy Systems in Turkey \cite{yasa}}
\label{price}
\end{table}
In addition to feed-in tariff, energy provider can benefit from additional incentives as long as some parts of the system is produced inside Turkey. For instance, by choosing a tower of a wind turbine which is a domestic production, an additional price is given to energy provider as local-bonus content. The local-bonus contents for wind turbines are listed in Table \ref{price2}.\par
\begin{table}[h!]
	\centering
	\begin{tabular}{cc}
		\hline
		\textbf{\begin{tabular}[c]{@{}c@{}}Local Content\\ for Wind Turbines\end{tabular}}   & \textbf{\begin{tabular}[c]{@{}c@{}}Local Content Incentive\\ (cent/kWh)\end{tabular}} \\ \hline
		Blade                                                                                & 0.8                                                                                   \\
		\begin{tabular}[c]{@{}c@{}}Generator and \\ Power Electronics\end{tabular}           & 1.0                                                                                   \\
		Turbine Tower                                                                        & 0.6                                                                                   \\
		\begin{tabular}[c]{@{}c@{}}All Mechanical Parts in \\ Rotor and Nacelle\end{tabular} & 1.3                                                                                   \\ \hline
	\end{tabular}
	\caption[Local Content Incentives for Wind Turbines]{Local Content Incentives for Wind Turbines\cite{yasa}}
	\label{price2}
\end{table}
With the increasing renewable energy penetration, the power system stability is getting more vulnerable to the disturbances. Since the grid operators are responsible for the successful operation of the grid, they have to work in order to improve the grid stability. Meantime, maintaining power system stability is not a responsibility of the energy providers. However, participation of the renewable energy providers is also a necessity to improve frequency stability of the grid. Hence, the grid operators have to come up with a solution to be implemented by the energy providers. Nonetheless, the solution should also be beneficial for the energy providers which are already satisfied with the existing purchase agreement with additional incentives. As long as the renewable energy providers are convinced to implement the solutions, the grid stability can be maintained against the increasing renewable penetration.
\section{Conclusion}
This chapter focuses on the frequency dynamics inside the power grid. The importance of the synchronous generators on the frequency is investigated by focusing on the inertial support behaviour. The characteristic of the inertial support is explained with the swing equation. The frequency behavior of the electricity grid is explored according to the supply and demand relation. \par
Grid frequency is maintained around the nominal frequency with the help of the frequency regulating mechanisms. Primary, secondary and tertiary frequency control mechanisms are briefly explained. In addition to technical details, the energy market is also summarized to present an economical perspective. Notice that economical aspect is also important for the feasibility of the inertial support that is expected to be implemented by energy providers.
% CHAPTER 1
\chapter{WIND TURBINE MODELLING}
\label{chp:3}

\section{VARIABLE SPEED PMSG WIND TURBINES}

The share of variable speed PMSG wind turbines is increasing worldwide due to the high efficiency and torque density. This type of wind turbines are equipped with full-scale power electronics which enable the turbine to have wide speed range. Even though the permanent magnet price fluctuates with time, the reliability and high efficiency of this type of turbine increase its share in the market.

 \begin{figure}[h!]
	\centering
	\includegraphics[width=1\linewidth]{Windmodel.png}
	\caption{Variable Speed Geared Wind Turbine Model}
	\label{varspeedpmsg}
\end{figure} 

Figure \ref{varspeedpmsg} shows the modelling of variable speed wind turbine. The aerodynamic sub-model includes turbine structure that captures power from the wind. The gearbox establishes the connection between wind turbine and PMSG. In this type of wind turbines, PMSG is not directly connected to grid so that the turbine speed is independent from the grid frequency. Therefore, back-to-back converter is used between generator and the electrical grid. The converter which is connected to PMSG is called Machine Side Converter (MSC) meanwhile the one connected to grid is called Grid Side Converter(GSC).

\subsection{Aerodynamic Model}
Aerodynamic model is the sub-model that captures power from the wind. The output of this block is the aerodynamic torque that rotates the turbine. However, the wind speed is not the only input. Turbine speed and pitch angle are also the inputs of the system since they affect the mechanical power that is captured from the wind.\par
The aerodynamic power of wind is given in Equation \ref{windpower} where $\rho_{air}$ is air density in $kg/cm^{3}$, $R$ is the blade radius in $m$ abd $v_{WIND}$ is the wind speed in $m/s$. Note that this is the available power of the air that is striking the turbine swept area and it is not possible to extract that amount of energy. Otherwise, the air would be standstill behind the wind turbine \cite{Ackermann2005a}.
\begin{equation}
P_{WIND}=0.5\rho_{air}\pi R^{2} v_{WIND}^{3}
\label{windpower}
\end{equation}
The wind turbine captures a fraction of the available wind power that is denominated as power coefficient $C_{p}$. Therefore, turbine power captured from wind can be found with the Equation \ref{turbinepower}.
\begin{equation}
P_{TUR}=C_{P}P_{WIND}
\label{turbinepower}
\end{equation}

Power coefficient determines the amount of power and it is a non-linear function of the tip speed ratio, $\lambda$ and pitch angle, $\beta$. Tip speed ratio is a parameter proportional with turbine speed. It can be defined as the ratio of the speed in the turbine tip to the wind speed as in the Equation \ref{tipspeed}. Power coefficient for a specific tip speed ratio and pitch angle can be found with the Equation \ref{cp} and \ref{lambdai} where $c_{1}$ is 0.5176, $c_{2}$ is 116, $c_{3}$ is 0.4, $c_{4}$ is 5, $c_{5}$ is 21 and $c_{6}$ is 0.0068 \cite{Heier}. \\
\begin{equation}
\lambda=\frac{\omega_{tur}R}{v_{WIND}}
\label{tipspeed}
\end{equation}
\begin{equation}
C_{p}(\lambda,\beta)=c_{1}(c_{2}/\lambda_{i}-c_{3}\beta-c_{4})e^{-c_{5}/\lambda{i}}+c_{6}\lambda
\label{cp}
\end{equation}
\begin{equation}
\frac{1}{\lambda_{i}}=\frac{1}{\lambda+0.08\beta}-\frac{0.035}{\beta^{3}+1} 
\label{lambdai}
\end{equation}

\begin{figure}[h!]
	\centering
	\includegraphics[width=.65\linewidth]{PowerCoefficient.png}
	\caption{Power Coefficient Variation with Tip Speed Ratio under Zero Pitch Angle}
	\label{variationofcp}
\end{figure} 
Variation of power coefficient $C_{p}$ is given in Figure \ref{variationofcp}. For the zero pitch angle, power coefficient has the maximum value of 0.48 for the tip speed ratio of 8.1. In order to ensure that the maximum of wind power is extracted, wind turbine should rotate a speed that gives the optimum tip speed ratio. 

\subsection{Gearbox}  
Variable speed PMSG wind turbines have a gearbox between turbine and generator except for direct-drive wind turbines. The gearbox increases angular speed and decreases the torque in the generator side.By decreasing the rated torque, generator size and cost can be reduced since the generator size is almost proportional to rated torque due to constant shear stress \cite{Polinder2013aa}.
\begin{figure}[h!]
	\centering
	\includegraphics[width=.45\linewidth]{gearbox.png}
	\caption{Gearbox Modelling}
	\label{gearboxmodel}
\end{figure}

\subsection{Permanent Magnet Synchronous Generator}  
% CHAPTER 1
\chapter{VALIDATION IN TEST CASE}
\label{chp:4}


\section{Örnek Kısım}
\label{sec:k1}

Kısıma yazılacaklar...

Bölüme referans vermek için \ref{chp:appendixA}
% CHAPTER 1
\chapter{VALIDATION IN TEST CASE}
\label{chp:5}
\section{P.M.Anderson 9 Bus Test Case}
\subsection{System Properties}
In order to understand frequency dynamics better, P.M. Anderson test case has been used in the study. The single line diagram of the system is given in Fig. \ref{ieee_9_bus}. The test case consists of three generators and three loads. Generators in the system are connected to 230 kV high voltage network with transformers.\par
\begin{figure}[h!]
	\centering
	\includegraphics[width=.85\linewidth]{ieee_9_bus.png}
	\caption{P.M.Anderson Test Case}
	\label{ieee_9_bus}
\end{figure}
The biggest generator in the system is a hydro power plant with a power rating of 247.5 MVA. The remaining ones are steam generators. The power ratings of the generators are given in Table \ref{generatorproperties}.\par
\begin{table}[h!]
	\centering
	\begin{tabular}{ccc}
		\hline
		\textbf{Generators} & \textbf{Power Rating (MVA)} & \textbf{Plant Type} \\ \hline
		Gen 1               & 247.5                       & Hydro				\\
		Gen 2               & 192                         & Steam               \\
		Gen 3               & 128                         & Steam               \\ \hline
	\end{tabular}
	\caption{Generator Properties of Test System}
	\label{generatorproperties}
\end{table}
The loads in the system are connected directly to the high voltage network. The active and reactive power ratings of the loads are listed in Table \ref{loadproperties}.
\begin{table}[h!]
	\centering
	\begin{tabular}{ccc}
		\hline
		\textbf{Generators} & \textbf{Active Power (MW)}  & \textbf{Reactive Power (MVAr)} \\ \hline
		Load A              & 125                      	  & 50				 \\
		Load B              & 90                          & 30                \\
		Load C              & 100                         & 35                \\ \hline
	\end{tabular}
	\caption{Load Properties of Test System}
	\label{loadproperties}
\end{table}
\subsection{Load Flow Analysis for Base Case}
Successful grid operation requires a load flow analysis in order to ensure that bus voltages are inside the allowed band and power flows are below the power carrying capabilities of the lines. Load flow results are given in Table \ref{loadflow_case1}.
\begin{table}[h!]
	\centering
	\begin{tabular}{cclccccc}
		\hline
		Bus \# & Bus Type & \multicolumn{1}{c}{Voltage} & Angle & Pg    & Qg     & Pl  & Ql \\ \hline
		1      & SL       & \multicolumn{1}{c}{1.04}    & 0     & 71.65 & 27.05  & 0   & 0  \\
		2      & PV       & \multicolumn{1}{c}{1.025}   & 9.28  & 163   & 6.65   & 0   & 0  \\
		3      & PV       & \multicolumn{1}{c}{1.025}   & 4.66  & 85    & -10.86 & 0   & 0  \\
		4      & PQ       & 1.0258                      & -2.22 & 0     & 0      & 0   & 0  \\
		5      & PQ       & 0.9956                      & -3.99 & 0     & 0      & 125 & 50 \\
		6      & PQ       & 1.0126                      & -3.69 & 0     & 0      & 90  & 30 \\
		7      & PQ       & 1.0258                      & 3.72  & 0     & 0      & 0   & 0  \\
		8      & PQ       & 1.0159                      & 0.73  & 0     & 0      & 100 & 35 \\
		9      & PQ       & 1.0323                      & 1.97  & 0     & 0      & 0   & 0  \\ \hline
	\end{tabular}
	\caption{Load Flow Results in Base Case}
	\label{loadflow_case1}
\end{table}
\subsection{Base Case Frequency Response for Additional Load Connection}
It is obvious that power system networks experience high RoCoF when either high amount of generation trips or high amount of load connects to system. These two main event can be used in the simulation to create frequency disturbances. Since the simulation in Simulink environment slows down with the increasing amount of generators, the disturbances are created with load connections.\par
\begin{table}[]
	\centering
	\begin{tabular}{ll}
		\hline
		Total System Load                      & 315 MW    \\
		Generator Droop Settings               & 5\%       \\
		Stored Kinetic Energy at Nominal Speed & 3.305 GWs \\
		Gen 1 Inertia Constant                 & 9.5515 s  \\
		Gen 2 Inertia Constant                 & 3.9216 s  \\
		Gen 3 Inertia Constant                 & 2.7665 s  \\ \hline
	\end{tabular}
	\caption{System Dynamical Properties}
	\label{systemdynamicaldata}
\end{table}
System dynamical properties are listed in Table \ref{systemdynamicaldata}. Power generation references are determined based on the load flow of powergui toolbox. Machine initialization toolbox is also used to initiate the state of generators in the system. However, the system does not start with the steady state. Still, system goes to steady state within a few seconds. Frequency of the network is disturbed with a load connection in the t=10 seconds in order to observe the frequency stability of the system. For 10\% load connection, a load of 31.5 MW is connected to system from Bus 6. Location of the additional load is depicted in Fig. \ref{ieee_9_bus_load}.\par
\begin{figure}[h!]
	\centering
	\includegraphics[width=.85\linewidth]{ieee_9_bus_load_position.png}
	\caption{Location of the Additional Load}
	\label{ieee_9_bus_load}
\end{figure}
According to the 10\% load connection to system, generator frequencies are shown in Fig. \ref{genfreqcase1}. Frequency of generator 1 is the most smooth one due to its huge inertia constant. Meanwhile, the generator 2 and generator 3 follow the frequency of generator 1 even though they oscillate with each other. \par
\begin{figure}[h!]
	\centering
	\includegraphics[width=.85\linewidth]{Case1_Generator_Freq.png}
	\caption{Generator Frequencies for 10\% Load Connection}
	\label{genfreqcase1}
\end{figure}
In the system, frequency of Bus 1 can be assumed as constant throughout the network since the system is small enough to assume a single frequency. This assumption can be verified by comparing the frequencies in Buses 1, 5 and 6. Fig. \ref{genfreqcase1_loadgen} shows the frequency of the generator 1 frequency as well as the load frequencies captured with Simulink PLL block. The only difference is the instant following the load connection. The sharp frequency decline delays the PLL loop to capture the frequnecy. 
\begin{figure}[h!]
	\centering
	\includegraphics[width=.85\linewidth]{Case1_Load_Gen_Freq.png}
	\caption{Frequencies in Generator 1, Load A and Load B}
	\label{genfreqcase1_loadgen}
\end{figure}
\section{Modified Case}
In this case, the P.M. Anderson test case is modified such that a wind farm consists of 20 wind turbine is connected to network. Wind farm is connected to Bus 5. Modified system is depicted in the Fig. \ref{ieee_9_bus_case2}. In this case, generator 2 and 3 are still assigned to same power values meanwhile generator 1 decreases its generation. 
\begin{figure}[h!]
	\centering
	\includegraphics[width=.85\linewidth]{ieee_9_bus_modified.png}
	\caption{Modified System Single Line Diagram}
	\label{ieee_9_bus_case2}
\end{figure}
\subsection{Load Flow Analysis for Modified Case}
Load flow analysis for modified case is listed in Table \ref{loadflow_case2}. The power injected from Bus 1 is decreased as expected. This can also be seen from the phase angle between 1 and 4. Phase angle difference between these buses decreased from $2.22^{\circ}$ to $1.18^{\circ}$.
\begin{table}[h!]
	\centering
	\begin{tabular}{cclccccc}
		\hline
		Bus \# & Bus Type & \multicolumn{1}{c}{Voltage} & Angle & Pg    & Qg     & Pl  & Ql \\ \hline
		1      & SL       & \multicolumn{1}{c}{1.04}    & 0     & 38.06 & 25.07  & 0   & 0  \\
		2      & PV       & \multicolumn{1}{c}{1.025}   & 11.33 & 163   & 6.65   & 0   & 0  \\
		3      & PV       & \multicolumn{1}{c}{1.025}   & 6.32  & 85    & -10.86 & 0   & 0  \\
		4      & PQ       & 1.0263                      & -1.18 & 0     & 0      & 0   & 0  \\
		5      & PQ       & 0.9995                      & -1.54 & 0     & 0      & 125 & 50 \\
		6      & PQ       & 1.0128                      & -2.43 & 0     & 0      & 90  & 30 \\
		7      & PQ       & 1.0266                      & 5.77  & 0     & 0      & 0   & 0  \\
		8      & PQ       & 1.0164                      & 2.62  & 0     & 0      & 100 & 35 \\
		9      & PQ       & 1.0326                      & 3.62  & 0     & 0      & 0   & 0  \\ \hline
	\end{tabular}
	\caption{Load Flow Results for Modified Case}
	\label{loadflow_case2}
\end{table}
\subsection{Modified Case Frequency Response for Additional Load Connection}
The modified base is very similar to the Base Case except for a wind farm located in Bus 5. The renewable energy system in this case can be considered as a negative load. Therefore, base case with decreased load is under discussion in this subsection. The same amount of load is taken into operation at Bus 6 and the frequency of the system is shown in Fig. \ref{Case1_2_freq}.\\
\begin{figure}[h!]
	\centering
	\includegraphics[width=.85\linewidth]{Case1_2.png}
	\caption{Comparison of Base Case and Modified Case Frequencies}
	\label{Case1_2_freq}
\end{figure}
Almost the same frequency response is observed in the system. The reason is that both systems have the same amount of stored kinetic energy. Another reason is the underutilization of the power system network. This can also be observed in the rate of change of frequencies in Fig. \ref{Case1_2_rocof}. Almost the same RoCoF values are observed in the system. This concludes that renewable energy penetration does not change frequency response of the system if the only change in the system is the inclusion of renewable energy system. Note that the renewable energy systems are intermittent energy sources. However, in this study,the source of the renewable energy system is assumed as constant. Therefore, the reason of frequency disturbance is load connection rather than the change in active power output of renewable systems.
\begin{figure}[h!]
	\centering
	\includegraphics[width=.85\linewidth]{Case1_2_rocof.png}
	\caption{Comparison of Base Case and Modified Case Frequencies}
	\label{Case1_2_rocof}
\end{figure}
\section{Decommissioned Case}
As seen in the Modified Case, the frequency response of the system does not change with renewable energy inclusion. However, it is inevitable that renewable energy systems will replace the conventional units in future. In this case, the smallest generator, generator 3, will be decommissioned. The decommissioned case diagram is shown in Fig. \ref{decommissioned_case}.\par
\begin{figure}[h!]
	\centering
	\includegraphics[width=.85\linewidth]{ieee_9_bus_decommissioned.png}
	\caption{Comparison of Base Case and Modified Case Frequencies}
	\label{decommissioned_case}
\end{figure}
Since the generator 3 is out of service, the stored kinetic energy is decreased in the system. Decommissioned system dynamical properties are updated and given in Table \ref{systemdynamicaldatacase3}.
\begin{table}[]
	\centering
	\begin{tabular}{ll}
		\hline
		Total System Load                      & 315 MW    \\
		Generator Droop Settings               & 5\%       \\
		Stored Kinetic Energy at Nominal Speed & 3.004 GWs \\
		Gen 1 Inertia Constant                 & 9.5515 s  \\
		Gen 2 Inertia Constant                 & 3.9216 s  \\
		\hline
	\end{tabular}
	\caption{System Dynamical Properties}
	\label{systemdynamicaldatacase3}
\end{table}
\subsection{Load Flow Analysis for Decommissioned Case}
Since the generator 3 is out of service, generator 1 loading will be increased. Load flow analysis for decommissioned case is given in Table \ref{loadflow_case3}.
\begin{table}[h!]
	\centering
	\begin{tabular}{cclccccc}
		\hline
		Bus \# & Bus Type & \multicolumn{1}{c}{Voltage} & Angle & Pg    & Qg     & Pl  & Ql \\ \hline
		1      & SL       & \multicolumn{1}{c}{1.04}    & 0     & 121.76& 16.26  & 0   & 0  \\
		2      & PV       & \multicolumn{1}{c}{1.025}   & 4.18  & 163   & 0.65   & 0   & 0  \\
		4      & PQ       & 1.0332                      & -3.74 & 0     & 0      & 0   & 0  \\
		5      & PQ       & 1.0083                      & -5.63 & 0     & 0      & 125 & 50 \\
		6      & PQ       & 1.0224                      & -7.65 & 0     & 0      & 90  & 30 \\
		7      & PQ       & 1.0294                      & -1.36 & 0     & 0      & 0   & 0  \\
		8      & PQ       & 1.0207                      & -5.82 & 0     & 0      & 100 & 35 \\
 \hline
	\end{tabular}
	\caption{Load Flow Results for Decommissioned Case}
	\label{loadflow_case3}
\end{table}
\subsection{Decommissioned Case Frequency Response for Additional Load Connection}
Same amount of additional load is taken into operation from Bus 6. System frequency response is observed and compared to Base Case and Modified Case in Fig. \ref{Case1_2_3_freq}. As soon from the figure, the frequency nadir decreased from 49.77Hz to 49.65Hz. This is due to the decrease in the stored kinetic energy in the system. Due to the decommission of generator 3, the frequency decreases steeper following to load connection.This can also be observed RoCoF comparison given in Fig. \ref{Case1_2_3_rocof}. 
\begin{figure}[h!]
	\centering
	\includegraphics[width=.85\linewidth]{Case1_2_3_freq.png}
	\caption{Comparison of Base Case,Modified Case and Decommissioned Case Frequency Responses}
	\label{Case1_2_3_freq}
\end{figure}
\begin{figure}[h!]
	\centering
	\includegraphics[width=.85\linewidth]{Case1_2_3_rocof.png}
	\caption{Comparison of Base Case,Modified Case and Decommissioned Case RoCoFs}
	\label{Case1_2_3_rocof}
\end{figure}

% CHAPTER 1
\chapter{EVALUATION OF FAST INERTIAL RESPONSE AND SYNTHETIC INERTIA IMPLEMENTATION}
\label{chp:6}



\section{FAST INERTIAL RESPONSE}

\section{SYNTHETIC INERTIA IMPLEMENTATION}

\section{FUTURE WORK}

\input{references.tex}

%
% References in Bibtex format goes into below indicated file with .bib extension
%\bibliography{thesis_references}
% You can use full name of authors, however most likely some of the Bibtex entries you will find, will use abbreviated first names
% If you don't want to correct each of them by hand, you can use abbreviated style for all of the references

%\bibliographystyle{abbrv}

% if you have more that one appendix, then use \appendices, otherwise use 
\appendices
\chapter{Wind Turbine Parameters}
\label{chp:appendixA}

\begin{table}[h]

		\centering
		\begin{tabular}{ccc}
			\hline
			\textbf{Property}       & \textbf{Value} & \textbf{Unit} \\ \hline
			Turbine Type            & GE2.75-103     & -             \\
			Rated Turbine Power     & 2.75           & MW            \\
			Converter Power Rating  & 3.04           & MVA           \\
			Rotor Diameter          & 103            & m             \\
			Blade Inertia           & 12600000       & $kgm^{2}$     \\
			Generator Speed Range   & 550-1735       & rpm           \\
			Rotor Speed Range       & 4.7-14.8       & rpm           \\
			Cut-in Wind Speed       & 3              & m/s           \\
			Cut-off Wind Speed      & 25             & m/s           \\
			Air Density             & 1.225          & $kg/m^{3}$    \\
			Gearbox Ratio           & 117.4          & -             \\
			Generator Rated Voltage & 690            & V             \\
			Generator Type          & PM Synchronous & -             \\
			Generator Inertia       & 240            & $kgm^{2}$     \\
			Generator Pole          & 4              & -             \\
			Generator Flux Linkage  & 2.5            & Vs            \\
			DC-Link Capacitance     & 27             & mF            \\
			DC-Link Voltage         & 1200           & V             \\ \hline
		\end{tabular}

	\caption{Wind Turbine Parameters Used in the Thesis}
\end{table}
\chapter{P.M. Anderson Test Case Properties}
\label{chp:appendixB}
\begin{table}[h]
	\centering
	\begin{tabular}{cccc}
		\hline
		\textbf{Loads} & \textbf{\begin{tabular}[c]{@{}c@{}}Location\\ (Bus Number)\end{tabular}} & \textbf{\begin{tabular}[c]{@{}c@{}}Active Power\\ (MW)\end{tabular}} & \textbf{\begin{tabular}[c]{@{}c@{}}Reactive Power\\ (MVAr)\end{tabular}} \\ \hline
		Load A         & 5                                                                        & 125                                                                  & 50                                                                       \\
		Load B         & 6                                                                        & 90                                                                   & 30                                                                       \\
		Load C         & 8                                                                        & 100                                                                  & 35                                                                       \\ \hline
	\end{tabular}
\caption{Load Data of the P.M. Anderson Test System}
\end{table}
\begin{table}[h]
	\centering
	\begin{tabular}{ccccc}
		\hline
		\textbf{From Bus} & \textbf{To Bus} & \textbf{\begin{tabular}[c]{@{}c@{}}Resistance (R)\\ (pu)\end{tabular}} & \textbf{\begin{tabular}[c]{@{}c@{}}Reactance (X)\\ (pu)\end{tabular}} & \textbf{\begin{tabular}[c]{@{}c@{}}Susceptance (B/2)\\ (pu)\end{tabular}} \\ \hline
		4                 & 5               & 0.0100                                                                 & 0.8500                                                                & 0.0880                                                                    \\
		4                 & 6               & 0.0170                                                                 & 0.0920                                                                & 0.0790                                                                    \\
		5                 & 7               & 0.0320                                                                 & 0.1610                                                                & 0.1530                                                                    \\
		6                 & 9               & 0.0390                                                                 & 0.1700                                                                & 0.1790                                                                    \\
		7                 & 8               & 0.0085                                                                 & 0.0720                                                                & 0.0745                                                                    \\
		8                 & 9               & 0.0119                                                                 & 0.1008                                                                & 0.1045                                                                    \\ \hline
	\end{tabular}
	\caption{Line Data of the P.M. Anderson Test System}
\end{table}
\begin{table}[h]
	\centering
	\begin{tabular}{cccc}
		\hline
		\textbf{Parameter}                                                             & \textbf{\begin{tabular}[c]{@{}c@{}}Location\\ (Bus Number)\end{tabular}} & \textbf{\begin{tabular}[c]{@{}c@{}}Active Power\\ (MW)\end{tabular}} & \textbf{\begin{tabular}[c]{@{}c@{}}Reactive Power\\ (MVAr)\end{tabular}} \\ \hline
		\begin{tabular}[c]{@{}c@{}}Nominal Apparent Power\\ (MVA)\end{tabular}         & 247.5                                                                    & 192                                                                  & 128                                                                      \\
		\begin{tabular}[c]{@{}c@{}}Nominal Voltage\\ (kV)\end{tabular}                 & 16.5                                                                     & 18                                                                   & 13.8                                                                     \\
		Nominal Power Factor                                                           & 1                                                                        & 0.85                                                                 & 0.85                                                                     \\
		Plant Type                                                                     & hydro                                                                    & steam                                                                & steam                                                                    \\
		Rotor Type                                                                     & salient                                                                  & round                                                                & round                                                                    \\
		\begin{tabular}[c]{@{}c@{}}Nominal Speed\\ (rpm)\end{tabular}                  & 180                                                                      & 3600                                                                 & 3600                                                                     \\
		\begin{tabular}[c]{@{}c@{}}$x_{d}$\\ (pu)\end{tabular}                         & 0.146                                                                    & 0.8958                                                               & 1.3125                                                                   \\
		\begin{tabular}[c]{@{}c@{}}$x_{d}'$\\ (pu)\end{tabular}                        & 0.0608                                                                   & 0.1198                                                               & 0.1813                                                                   \\
		\begin{tabular}[c]{@{}c@{}}$x_{q}$\\ (pu)\end{tabular}                         & 0.0969                                                                   & 0.8645                                                               & 1.2578                                                                   \\
		\begin{tabular}[c]{@{}c@{}}$x_{q}'$\\ (pu)\end{tabular}                        & 0.0969                                                                   & 0.1969                                                               & 0.25                                                                     \\
		\begin{tabular}[c]{@{}c@{}}$x_{l}$\\ (pu)\end{tabular}                         & 0.0336                                                                   & 0.0521                                                               & 0.0742                                                                   \\
		\begin{tabular}[c]{@{}c@{}}$\tau_{d0}'$\\ (s)\end{tabular}                     & 8.96                                                                     & 6                                                                    & 5.89                                                                     \\
		\begin{tabular}[c]{@{}c@{}}$\tau_{q0}'$\\ (s)\end{tabular}                     & 0                                                                        & 0.535                                                                & 0.6                                                                      \\
		\begin{tabular}[c]{@{}c@{}}Stored Energy at nominal speed\\ (MWs)\end{tabular} & 2364                                                                     & 640                                                                  & 301                                                                      \\
		\begin{tabular}[c]{@{}c@{}}Inertia\\ (s)\end{tabular}                          & 9.5515                                                                   & 3.3333                                                               & 2.3516                                                                   \\ \hline
	\end{tabular}
	\caption{Generator Data of the P.M. Anderson Test System}
\end{table}
\begin{table}[h]
	\centering
	\begin{tabular}{cccccc}
		\hline
		\textbf{Transformers} & \textbf{From Bus} & \textbf{To Bus} & \textbf{\begin{tabular}[c]{@{}c@{}}High Voltage Side\\ (kV)\end{tabular}} & \textbf{\begin{tabular}[c]{@{}c@{}}Low Voltage Side\\ (kV)\end{tabular}} & \textbf{\begin{tabular}[c]{@{}c@{}}Reactance\\ (pu)\end{tabular}} \\ \hline
		Transformer 1         & 1                 & 4               & 230                                                                       & 16.5                                                                     & 0.0576                                                            \\
		Transformer 1         & 2                 & 7               & 230                                                                       & 18                                                                       & 0.0625                                                            \\
		Transformer 1         & 3                 & 9               & 230                                                                       & 13.8                                                                     & 0.0586                                                            \\ \hline
	\end{tabular}
\caption{Transformer Data of the P.M. Anderson Test System}
\end{table}
\end{document}
